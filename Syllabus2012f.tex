\documentclass[]{article}
\usepackage[T1]{fontenc}
\usepackage{lmodern}
\usepackage{amssymb,amsmath}
\usepackage{ifxetex,ifluatex}
\usepackage{enumitem}
\setlist{nolistsep}

\topmargin=-0.45in      %
\evensidemargin=0in     %
\oddsidemargin=0in      %
\textwidth=6.5in        %
\textheight=9.0in       %
\headsep=0.25in         %

\usepackage{fixltx2e} % provides \textsubscript
% use microtype if available
\IfFileExists{microtype.sty}{\usepackage{microtype}}{}
\ifnum 0\ifxetex 1\fi\ifluatex 1\fi=0 % if pdftex
  \usepackage[utf8]{inputenc}
\else % if luatex or xelatex
  \usepackage{fontspec}
  \ifxetex
    \usepackage{xltxtra,xunicode}
  \fi
  \defaultfontfeatures{Mapping=tex-text,Scale=MatchLowercase}
  \newcommand{\euro}{€}
\fi
\ifxetex
  \usepackage[setpagesize=false, % page size defined by xetex
              unicode=false, % unicode breaks when used with xetex
              xetex]{hyperref}
\else
  \usepackage[unicode=true]{hyperref}
\fi
\hypersetup{breaklinks=true,
            bookmarks=true,
            pdfauthor={},
            pdftitle={ASTR 2600 Computational Techniques},
            colorlinks=true,
            urlcolor=blue,
            linkcolor=magenta,
            pdfborder={0 0 0}}
\setlength{\parindent}{0pt}
\setlength{\parskip}{6pt plus 2pt minus 1pt}
\setlength{\emergencystretch}{3em}  % prevent overfull lines
\setcounter{secnumdepth}{0}

\title{ASTR 2600 Computational Techniques}
\author{}
\date{}

\begin{document}
%\maketitle

\begin{center} \textbf{\large ASTR 2600~ Computational Techniques}

{ Fall 2012~ Syllabus}
\end{center}


Instructor: Adam Ginsburg (\url{adam.ginsburg@colorado.edu}) \\
Class e-mail address: \url{ASTR2600@gmail.com} (feel free to add on gchat / Google Talk)
%``mailto:Adam.Ginsburg@colorado.edu'' Adam.Ginsburg@colorado.edu

\textbf{Office (lab) hours}: 
    %Tuesday 3-4 in the Cosmos lab
    Thursday 5:15-6:15pm (immediately after class)

All other times: Adam's office is Duane D217

\textbf{Student Assistant}: Cameron Wedge (\url{cameron.wedge@colorado.edu})

\textbf{Meeting location}: Sommers-Bausch Observatory classroom

\textbf{Meeting time}: 4pm-5:15pm Tuesdays \& Thursdays

\textbf{Required Text}: \emph{Computer Programming for Scientists Using IDL} Parts 0-3
by Dewey Anderson.  The current version is different from the text used last
semester.  The text will appear in the bookstore in 2 ``volumes''.  Currently
the first volume containing Part 0 \& Part 1 is available in the bookstore.

% \textbf{Suggested Text}: \emph{Learning the Unix Operating System} by Peek,
% Todino \& Strang


\textbf{Purpose:} This is a course intended to teach you the basics of
computer programming using the IDL language and introduce you to the
Unix operating system.~ We will be using Sunray terminals on the cosmos
computer in the Cosmos Lab at the Sommers-Bausch Observatory.


\textbf{Assignments:} Programming exercises and homeworks will be done
on the Cosmos Lab computers.~ This is not a lab class so expect to do
the assignments outside of class time.~ Cameron and I will be available
to help you in the lab during our scheduled office hours.~ If you need
access to the building after hours, we will have Keith Gleason (SBO
manager) get your CU ID card approved for entry.


Assignments will be turned in by copying the appropriate files into
a directory of the form:


\verb|/home/shared/astr2600/assignment1/YourName/|


where ``assignment1'' will be replaced with the appropriate assignment number,
e.g. ``assignment2'' and YourName is your first and last name written in that
fashion, e.g., AdamGinsburg.~ You each will have your own personal directories
for turning in files and there's a separate directory for each assignment.  If
you would prefer a directory name using a shortened or alternative version of
your first name, please let me know and I will change the directory names.


Most assignments are in the book.~ You should prepend your name to the
beginning of every filename of the files you turn in, e.g.
\verb|YourName_ex2.pro|.


\textbf{Books:} The IDL text is required and there will be reading assignments
given each class.~ You are expected to do the reading \emph{before} the
next class.  There may be clicker questions in class based on the assigned
reading.


\textbf{Grading:}

Assignments will be given each week.~ As a rule they will consist of 3
parts: an exercise, a ``Whuduzitdo?'' (What does it do?) and a ``Graded
Homework''. I plan to assign them on Tuesdays. The exercises and
Whuduzitdo's will be due \emph{before the next class} (Thursday).~ The
homework will be due the following Tuesday (by end of the day).


The exercise part is mostly ``type this in and see what happens'' to
show you IDL.~ (Don't just treat it as a typing exercise.~ Pay attention
to what you're typing and what it's doing.)~ The Whuduzitdo section
shows you IDL statements or programs and \emph{you} have to say what it
does, i.e. you pretend to be the computer.~ This usually means answering
a question like ``What does it print?'' or ``What's the value of this
variable?''.~ Sometimes it explains an objective of the IDL and you have
to say why it doesn't meet that objective and how you would fix it.~
This is important practice for debugging (finding your mistakes in)
programs.


The exercise and the Whuduzitdo will each be given a grade of 0, 1 or 2
points. ~

~ 2 points if you do the entire thing

~ 1 point if it seems a poor or partial effort in some fashion \emph{or
is turned in late}.

~ 0 points if you don't do it, or do so little of it that I can't
justify giving you 1 point

The total grade for all exercises \& Whuduzitdo's combined will count as
15\% of your grade.

\emph{No assignment will be accepted more than 2 weeks late.}


Notice that the correctness of answers to the Whuduzitdo's don't count
toward the grade.~ The objective is to get you to \emph{try} and figure
it out, no worrying about whether you got it right.~ It's a learning
exercise, not a test of your knowledge.


Homeworks will be graded on a 0-100\% scale. There is a late penalty of
3 points per day.

Weekends and holidays  do not count in the late
penalty assessment, e.g., a homework due at midnight on Tuesday night
will take a 3 point hit if you turn it in on Wednesday, a 12 point hit if
you turn it on Friday and a 15 point hit if you turn it on Saturday,
Sunday or Monday.

I will waive your largest homework late penalty for the semester.


There \emph{may} be one exam that would consist almost entirely of
things like you find in the Whuduzitdo's.~ It would count the same as a
single homework.

Letter grades are assigned in 10-point bands, i.e., A's are
90-100, B's 80-89, etc.


\textbf{Advice:} You can usually tell if your homework is working.~
\emph{Don't} turn in homework assignments that don't work.~ I don't give
much credit for something that doesn't work.~ Take the 3-point late
penalty and get help the next day.~ That's why the late penalty is as
small as it is.~ But BEWARE: Don't habitually turn in late assignments.~
The late penalties will build up faster than you think.~ It's a shame to
find your A-level work earning you a grade of C (or worse).~ Many of the
assignments build on earlier assignments so don't plan on ``skipping''
an assignment to turn it later.~ Getting behind
in this class will snowball on you.~ Stay caught up.

\textbf{Clickers (and attendance):}  Clickers may be used in class.  These may
be survey questions or “What does it do?” questions.  10\% of your grade will
be based on clicker responses.  Most of this grade will be participation,
though a small portion will be correctness when there is a right answer.
Attendance is required to get the clicker credit.  Excused absences will not
count against your grade.

\begin{tabular}[h]{|cc|}
    Class Component & Grade Percentage \\
    Exercises and Whuduzitdo?s & 15\% \\
    Clickers and Attendance & 10\% \\
    Homework & 75\% \\
\end{tabular}



\textbf{Rough course outline:}

How computers work (Chapter 0) will be mixed in to the class. 

Weeks 1-5: Introduction

\begin{itemize}
    \item Unix
    \item Interactive IDL:  Chapters 1 - 9
        \begin{itemize}
            \item ~ Data types, equations, built-in procedures \& functions, graphics
        \end{itemize}
    \item Assignments dealing with: calculations, reading/writing files,
displaying data (plotting), images
\end{itemize}


Weeks 6-10:~ Programming in IDL~

\begin{itemize}
\item writing IDL scripts \& programs~ Chapters 10, 11

\item flow control constructs, IF, FOR, WHILE~ ~ Ch 12

\item writing your own procedures \& functions ~ Ch 13

\item Software design Ch 14

\item data structures~ Ch 15

\item Animation~ Ch 16

\item Assignments dealing with dynamical simulations, differential
equations, N-body simulation
\end{itemize}


%Fall break (12 weeks before, 3 weeks after)


Weeks 11-15:~ More-advanced techniques

\begin{itemize}
\item Random numbers, interpolation, curve-fitting

\item Object-oriented programming and ``classes''

\item Recursion:~ factorial, binary tree, Measuring stars in telescope
images

\item An introduction to Fortran and/or Python
\end{itemize}






\textbf{To activate your cosmos account:} Get on a computer on campus
(not from home) and use a web browser and go to


\url{https://sac.colorado.edu/}


There you can use the drop down menu to select ``cosmos'', enter your CU
login name and IdentiKey Password and activate your account.~ You should
then be able to login to the cosmos computers in the Cosmos Lab.


Standard Caveat: All aspects of this syllabus are subject to change.

\textbf{HONOR CODE}


As a CU student, you are required to be familiar with CU's honor code
and to not violate it.


This class will require some special care because the programming
environment often has a collaborative feel.~ Indeed, much of my
programming expertise comes from seeing how others program and
incorporating the good ideas I've seen.~ But the work-world is not the
school-world and what is encouraged there is often forbidden here. ~


Under no circumstances should any computer files be electronically
shared without explicit permission from me (e.g., someone brings in a
neat data file we all want to analyze). ~


Talking with others about the \emph{exercises} is encouraged even while
you are doing them.


Don't just ask other people the answer to the Whuduzitdo's.~ Try and
figure it out.~ It's a learning experience.~ That's why
\emph{correctness} of your answer doesn't count toward the grade.


Homeworks should be entirely your own work.  You may not work in groups and you
may not share code.






\textbf{The Boulder Provost's Disability Task Force recommended syllabus statement:}

(1)  If you qualify for accommodations because of a disability, please submit to
your professor a letter from Disability Services in a timely manner (for exam
accommodations provide your letter at least one week prior to the exam) so that
your needs can be addressed. Disability Services determines accommodations
based on documented disabilities. Contact Disability Services at 303-492-8671
or by e-mail at dsinfo@colorado.edu.

If you have a temporary medical condition or injury, see Temporary Medical
Conditions: Injuries, Surgeries, and Illnesses guidelines under Quick Links at
Disability Services website and discuss your needs with your professor.

(2) It is the responsibility of every instructor to clearly explain his or her
procedures about absences due to religious observances in the course syllabus
so that all students are fully informed, in writing, near the beginning of each
semester's classes.  Campus policy regarding religious observances states
that faculty must make reasonable accommodation for them and in so doing, be
careful not to inhibit or penalize those students who are exercising their
rights to religious observance. Faculty should be aware that a given religious
holiday may be observed with very different levels of attentiveness by
different members of the same religious group and thus may require careful
consideration to the particulars of each individual case.  See
\url{http://www.colorado.edu/policies/fac_relig.html}

If you have questions about providing students with religious accommodations,
please contact the Office of Discrimination and Harassment at 303-492-2797.

A comprehensive calendar of the religious holidays most commonly observed by
CU-Boulder students is at \url{http://www.interfaithcalendar.org/}

Recommended syllabus statement:

Campus policy regarding religious observances requires that faculty make every
effort to deal reasonably and fairly with all students who, because of
religious obligations, have conflicts with scheduled exams, assignments or
required attendance.  In this class, up to three missed classes will be excused
for religions obligations.  The 3-points-per-day “late fee” will not be assessed
for excused absences, but the 2-week deadline will still be enforced.
See full details at \url{http://www.colorado.edu/policies/fac_relig.html}

(3)  Faculty and students should be aware of the campus “Classroom
Behavior” policy at
\url{http://www.colorado.edu/policies/classbehavior.html} as well as faculty rights
and responsibilities listed at
\url{http://www.colorado.edu/FacultyStaff/faculty-booklet.html#Part_1}
These documents describe examples of unacceptable classroom behavior and
provide information on how to handle such circumstances should they arise.
Faculty are encouraged to address the issue of classroom behavior in the
syllabus.

Recommended syllabus statement:

Students and faculty each have responsibility for maintaining an appropriate
learning environment. Those who fail to adhere to such behavioral standards may
be subject to discipline. Professional courtesy and sensitivity are especially
important with respect to individuals and topics dealing with differences of
race, color, culture, religion, creed, politics, veteran's status, sexual
orientation, gender, gender identity and gender expression, age, disability,
and nationalities.  Class rosters are provided to the instructor with the
student's legal name. I will gladly honor your request to address you by an
alternate name or gender pronoun. Please advise me of this preference early in
the semester so that I may make appropriate changes to my records.  See policies at
\url{http://www.colorado.edu/policies/classbehavior.html}   and at
\url{http://www.colorado.edu/studentaffairs/judicialaffairs/code.html#student_code}



(4)  The Office of Discrimination and Harassment recommends the following
syllabus statement:

The University of Colorado Boulder (CU-Boulder) is committed to maintaining a
positive learning, working, and living environment. The University of Colorado
does not discriminate on the basis of race, color, national origin, sex, age,
disability, creed, religion, sexual orientation, or veteran status in admission
and access to, and treatment and employment in, its educational programs and
activities. (Regent Law, Article 10, amended 11/8/2001).  CU-Boulder will not
tolerate acts of discrimination or harassment based upon Protected Classes or
related retaliation against or by any employee or student. For purposes of this
CU-Boulder policy, “Protected Classes” refers to race, color, national origin,
sex, pregnancy,  age, disability, creed, religion, sexual orientation, gender
identity, gender expression,  or veteran status.  Individuals who believe they
have been discriminated against should contact the Office of Discrimination and
Harassment (ODH) at 303-492-2127 or the Office of Student Conduct (OSC) at
303-492-5550.  Information about the ODH, the above referenced policies, and
the campus resources available to assist individuals regarding discrimination
or harassment can be obtained at \url{http://www.colorado.edu/odh}

(5)  The Boulder campus has a student Honor Code and individual faculty members
are expected to familiarize themselves with its tenets and follow the approved
procedures should violations be perceived.  The Honor Council recommended
syllabus statement:

All students of the University of Colorado at Boulder are responsible for
knowing and adhering to the academic integrity policy of this institution.
Violations of this policy may include: cheating, plagiarism, aid of academic
dishonesty, fabrication, lying, bribery, and threatening behavior.  All
incidents of academic misconduct shall be reported to the Honor Code Council
(honor@colorado.edu; 303-735-2273). Students who are found to be in violation
of the academic integrity policy will be subject to both academic sanctions
from the faculty member and non-academic sanctions (including but not limited
to university probation, suspension, or expulsion). Other information on the
Honor Code can be found at \url{http://www.colorado.edu/policies/honor.html}  and at
\url{http://www.colorado.edu/academics/honorcode/}

\end{document}
