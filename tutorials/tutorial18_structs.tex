%\documentclass[11pt,letterpaper,notitlepage,onesided]{tex/nwh_hw}
%\documentclass[11pt,letterpaper,notitlepage]{article}
\documentclass{article}
% Change "article" to "report" to get rid of page number on title page
\usepackage{amsmath,amsfonts,amsthm,amssymb}
\usepackage{setspace}
\usepackage{textcomp}
\usepackage{listings}
\lstset{basicstyle=\ttfamily} % <<< This line added
\lstset{upquote=true}
\lstset{breakatwhitespace=true}
\renewcommand{\ttdefault}{cmtt}

\usepackage{Tabbing}
\usepackage{textcomp}
\usepackage{fancyhdr}
\usepackage{lastpage}
\usepackage{extramarks}
\usepackage{chngpage}
\usepackage{soul,color}
\usepackage{graphicx,float,wrapfig}
\usepackage{parskip}
\usepackage[utf8]{inputenc}
\usepackage[T1]{fontenc}

% In case you need to adjust margins:
\topmargin=-0.45in      %
\evensidemargin=0in     %
\oddsidemargin=0in      %
\textwidth=6.5in        %
\textheight=9.0in       %
\headsep=0.25in         %

% Homework Specific Information
\newcommand{\hmwkTitle}{Tutorial: Structures and Pointers, Linked Lists}
\newcommand{\hmwkDueDate}{DATE, 4:00 PM}
\newcommand{\hmwkClass}{ASTR 2600}
\newcommand{\hmwkClassTime}{4:00 PM T/Th}
\newcommand{\hmwkClassInstructor}{Adam Ginsburg}
\newcommand{\hmwkAuthorName}{Dewey Anderson}

% Setup the header and footer
\pagestyle{fancy}                                                       %
%\lhead{\hmwkAuthorName}                                                 %
\chead{\hmwkClass\: \hmwkTitle}  %
\rhead{\firstxmark}                                                     %
\lfoot{\lastxmark}                                                      %
\cfoot{}                                                                %
\rfoot{Page\ \thepage\ of\ \pageref{LastPage}}                          %
\renewcommand\headrulewidth{0.4pt}                                      %
\renewcommand\footrulewidth{0.4pt}                                      %

\usepackage[utf8]{inputenc}
\usepackage[unicode=true]{hyperref}
\hypersetup{breaklinks=true,
            bookmarks=true,
            pdfauthor={},
            pdftitle={Connecting to the cosmos computer from home using Microsoft Windows},
            colorlinks=true,
            urlcolor=blue,
            linkcolor=magenta,
            pdfborder={0 0 0}}

% This is used to trace down (pin point) problems
% in latexing a document:
%\tracingall

%%%%%%%%%%%%%%%%%%%%%%%%%%%%%%%%%%%%%%%%%%%%%%%%%%%%%%%%%%%%%
% Some tools
\newcommand{\enterProblemHeader}[1]{\nobreak\extramarks{#1}{#1 continued on next page\ldots}\nobreak%
                                    \nobreak\extramarks{#1 (continued)}{#1 continued on next page\ldots}\nobreak}%
\newcommand{\exitProblemHeader}[1]{\nobreak\extramarks{#1 (continued)}{#1 continued on next page\ldots}\nobreak%
                                   \nobreak\extramarks{#1}{}\nobreak}%

\newlength{\labelLength}
\newcommand{\labelAnswer}[2]
  {\settowidth{\labelLength}{#1}%
   \addtolength{\labelLength}{0.25in}%
   \changetext{}{-\labelLength}{}{}{}%
   \noindent\fbox{\begin{minipage}[c]{\columnwidth}#2\end{minipage}}%
   \marginpar{\fbox{#1}}%

   % We put the blank space above in order to make sure this
   % \marginpar gets correctly placed.
   \changetext{}{+\labelLength}{}{}{}}%

\setcounter{secnumdepth}{0}
\newcommand{\homeworkProblemName}{}%
\newcounter{homeworkProblemCounter}%
\newenvironment{homeworkProblem}[1][Problem \arabic{homeworkProblemCounter}]%
  {\stepcounter{homeworkProblemCounter}%
   \renewcommand{\homeworkProblemName}{#1}%
   \section{\homeworkProblemName}%
   \enterProblemHeader{\homeworkProblemName}}%
  {\exitProblemHeader{\homeworkProblemName}}%

\newcommand{\problemAnswer}[1]
  {\noindent\fbox{\begin{minipage}[c]{\columnwidth}#1\end{minipage}}}%

\newcommand{\problemLAnswer}[1]
  {\labelAnswer{\homeworkProblemName}{#1}}

\newcommand{\homeworkSectionName}{}%
\newlength{\homeworkSectionLabelLength}{}%
\newenvironment{homeworkSection}[1]%
  {% We put this space here to make sure we're not connected to the above.
   % Otherwise the changetext can do funny things to the other margin

   \renewcommand{\homeworkSectionName}{#1}%
   \settowidth{\homeworkSectionLabelLength}{\homeworkSectionName}%
   \addtolength{\homeworkSectionLabelLength}{0.25in}%
   \changetext{}{-\homeworkSectionLabelLength}{}{}{}%
   \subsection{\homeworkSectionName}%
   \enterProblemHeader{\homeworkProblemName\ [\homeworkSectionName]}}%
  {\enterProblemHeader{\homeworkProblemName}%

   % We put the blank space above in order to make sure this margin
   % change doesn't happen too soon (otherwise \sectionAnswer's can
   % get ugly about their \marginpar placement.
   \changetext{}{+\homeworkSectionLabelLength}{}{}{}}%

\newcommand{\sectionAnswer}[1]
  {% We put this space here to make sure we're disconnected from the previous
   % passage

   \noindent\fbox{\begin{minipage}[c]{\columnwidth}#1\end{minipage}}%
   \enterProblemHeader{\homeworkProblemName}\exitProblemHeader{\homeworkProblemName}%
   \marginpar{\fbox{\homeworkSectionName}}%

   % We put the blank space above in order to make sure this
   % \marginpar gets correctly placed.
   }%

%%%%%%%%%%%%%%%%%%%%%%%%%%%%%%%%%%%%%%%%%%%%%%%%%%%%%%%%%%%%%


%%%%%%%%%%%%%%%%%%%%%%%%%%%%%%%%%%%%%%%%%%%%%%%%%%%%%%%%%%%%%
% Make title
\title{\vspace{2in}\textmd{\textbf{\hmwkClass:\ \hmwkTitle}}\\\normalsize\vspace{0.1in}\small{Due\ on\ \hmwkDueDate}\\\vspace{0.1in}\large{}\vspace{3in}}
\date{}
%\author{\textbf{\hmwkAuthorName}}
%%%%%%%%%%%%%%%%%%%%%%%%%%%%%%%%%%%%%%%%%%%%%%%%%%%%%%%%%%%%%

\begin{document}
\begin{spacing}{1.0}
%\maketitle
\newpage



\section{Structures with Pointers}

Open your \verb|CarStruct__DEFINE.pro| from Tutorial 16.  
Edit it to include a new \emph{null} pointer for the entry \verb|serviced_at_miles|.

\section{Linked Lists}
\subsection{1. Get Adam's code} \\
First, get the procedures I defined.  

\begin{lstlisting}
git pull
git commit -a
git fetch upstream
git rebase upstream/master
git push
\end{lstlisting}

You should have the following 4 procedures in your \verb|Tutorial18_LinkedLists| directory:
\begin{lstlisting}
insert_ll.pro
node__DEFINE.pro
print_ll.pro
test_insert.pro
\end{lstlisting}    

Copy these into your \verb|ASTR2600/tutorials| directory:\\
\verb|cp -r Tutorial18_LinkedLists ~/ASTR2600/tutorials|\\
Then make sure you add the new directory to your repository and push it:
\verb|cd ~/ASTR2600/tutorials/|\\
\verb|git add Tutoria18_LinkedLists|\\
\verb|git commit|\\
\verb|git push|\\
(obviously these commands won't work if you're in the wrong directory)

Examine these files - open them all in \verb|gedit|.  In particular, look at
\verb|test_insert.pro| to see how to use \verb|insert_ll| and \verb|print_ll|.

In order to run this code, you'll probably need to add
\verb|Tutorial18_LinkedLists| to your \verb|!PATH|.\\
\verb|IDL> !PATH=!PATH+":/full/path/to/Tutorial18_LinkedLists"|\\
(you have to replace \verb|/full/path/to| with the correct full path) \\


\subsection{2. New code} 

Create three new procedures:\\
\verb|remove_node.pro|\\
\verb|add_tail.pro|\\
\verb|add_head.pro|

These procedures should all accept a \emph{pointer to a node} and some integer data.

Also, create a function: \\
\verb|n_elements_ll.pro|\\
that will return the number of elements in the linked list.

Here's an example node structure:\\
\verb?[1|next] -> [2|next] -> [3|next] -> !null?

If you run the command 
\verb|remove_node,2|\\ 
the resulting structure should be\\
\verb?[1|next] -> [3|next] -> !null?

If you then run 
\verb|add_tail,4|\\
you should get \\
\verb?[1|next] -> [3|next] -> [4|next] -> !null?

If you then run 
\verb|add_head,0|\\
you should get \\
\verb?[0|next] -> [1|next] -> [3|next] -> [4|next] -> !null?\\

Remember that the “standard” loop through a list should include something like:
\begin{lstlisting}
while (node ne !null) do begin
    ; something (possibly an if X then break statement)
    node = (*node).next
endwhile
\end{lstlisting}

\end{spacing}
\end{document}

