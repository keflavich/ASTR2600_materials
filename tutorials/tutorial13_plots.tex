%\documentclass[11pt,letterpaper,notitlepage,onesided]{tex/nwh_hw}
%\documentclass[11pt,letterpaper,notitlepage]{article}
\documentclass{article}
% Change "article" to "report" to get rid of page number on title page
\usepackage{amsmath,amsfonts,amsthm,amssymb}
\usepackage{setspace}
\usepackage{textcomp}
\usepackage{listings}
\lstset{basicstyle=\ttfamily} % <<< This line added
\lstset{upquote=true}
\lstset{breakatwhitespace=true}
\renewcommand{\ttdefault}{cmtt}

\usepackage{Tabbing}
\usepackage{textcomp}
\usepackage{fancyhdr}
\usepackage{lastpage}
\usepackage{extramarks}
\usepackage{chngpage}
\usepackage{soul,color}
\usepackage{graphicx,float,wrapfig}
\usepackage{parskip}
\usepackage[utf8]{inputenc}
\usepackage[T1]{fontenc}

% In case you need to adjust margins:
\topmargin=-0.45in      %
\evensidemargin=0in     %
\oddsidemargin=0in      %
\textwidth=6.5in        %
\textheight=9.0in       %
\headsep=0.25in         %

% Homework Specific Information
\newcommand{\hmwkTitle}{Tutorial: Plots and Color}
\newcommand{\hmwkDueDate}{DATE, 4:00 PM}
\newcommand{\hmwkClass}{ASTR 2600}
\newcommand{\hmwkClassTime}{4:00 PM T/Th}
\newcommand{\hmwkClassInstructor}{Adam Ginsburg}
\newcommand{\hmwkAuthorName}{Dewey Anderson}

% Setup the header and footer
\pagestyle{fancy}                                                       %
%\lhead{\hmwkAuthorName}                                                 %
\chead{\hmwkClass\: \hmwkTitle}  %
\rhead{\firstxmark}                                                     %
\lfoot{\lastxmark}                                                      %
\cfoot{}                                                                %
\rfoot{Page\ \thepage\ of\ \pageref{LastPage}}                          %
\renewcommand\headrulewidth{0.4pt}                                      %
\renewcommand\footrulewidth{0.4pt}                                      %

\usepackage[utf8]{inputenc}
\usepackage[unicode=true]{hyperref}
\hypersetup{breaklinks=true,
            bookmarks=true,
            pdfauthor={},
            pdftitle={Connecting to the cosmos computer from home using Microsoft Windows},
            colorlinks=true,
            urlcolor=blue,
            linkcolor=magenta,
            pdfborder={0 0 0}}

% This is used to trace down (pin point) problems
% in latexing a document:
%\tracingall

%%%%%%%%%%%%%%%%%%%%%%%%%%%%%%%%%%%%%%%%%%%%%%%%%%%%%%%%%%%%%
% Some tools
\newcommand{\enterProblemHeader}[1]{\nobreak\extramarks{#1}{#1 continued on next page\ldots}\nobreak%
                                    \nobreak\extramarks{#1 (continued)}{#1 continued on next page\ldots}\nobreak}%
\newcommand{\exitProblemHeader}[1]{\nobreak\extramarks{#1 (continued)}{#1 continued on next page\ldots}\nobreak%
                                   \nobreak\extramarks{#1}{}\nobreak}%

\newlength{\labelLength}
\newcommand{\labelAnswer}[2]
  {\settowidth{\labelLength}{#1}%
   \addtolength{\labelLength}{0.25in}%
   \changetext{}{-\labelLength}{}{}{}%
   \noindent\fbox{\begin{minipage}[c]{\columnwidth}#2\end{minipage}}%
   \marginpar{\fbox{#1}}%

   % We put the blank space above in order to make sure this
   % \marginpar gets correctly placed.
   \changetext{}{+\labelLength}{}{}{}}%

\setcounter{secnumdepth}{0}
\newcommand{\homeworkProblemName}{}%
\newcounter{homeworkProblemCounter}%
\newenvironment{homeworkProblem}[1][Problem \arabic{homeworkProblemCounter}]%
  {\stepcounter{homeworkProblemCounter}%
   \renewcommand{\homeworkProblemName}{#1}%
   \section{\homeworkProblemName}%
   \enterProblemHeader{\homeworkProblemName}}%
  {\exitProblemHeader{\homeworkProblemName}}%

\newcommand{\problemAnswer}[1]
  {\noindent\fbox{\begin{minipage}[c]{\columnwidth}#1\end{minipage}}}%

\newcommand{\problemLAnswer}[1]
  {\labelAnswer{\homeworkProblemName}{#1}}

\newcommand{\homeworkSectionName}{}%
\newlength{\homeworkSectionLabelLength}{}%
\newenvironment{homeworkSection}[1]%
  {% We put this space here to make sure we're not connected to the above.
   % Otherwise the changetext can do funny things to the other margin

   \renewcommand{\homeworkSectionName}{#1}%
   \settowidth{\homeworkSectionLabelLength}{\homeworkSectionName}%
   \addtolength{\homeworkSectionLabelLength}{0.25in}%
   \changetext{}{-\homeworkSectionLabelLength}{}{}{}%
   \subsection{\homeworkSectionName}%
   \enterProblemHeader{\homeworkProblemName\ [\homeworkSectionName]}}%
  {\enterProblemHeader{\homeworkProblemName}%

   % We put the blank space above in order to make sure this margin
   % change doesn't happen too soon (otherwise \sectionAnswer's can
   % get ugly about their \marginpar placement.
   \changetext{}{+\homeworkSectionLabelLength}{}{}{}}%

\newcommand{\sectionAnswer}[1]
  {% We put this space here to make sure we're disconnected from the previous
   % passage

   \noindent\fbox{\begin{minipage}[c]{\columnwidth}#1\end{minipage}}%
   \enterProblemHeader{\homeworkProblemName}\exitProblemHeader{\homeworkProblemName}%
   \marginpar{\fbox{\homeworkSectionName}}%

   % We put the blank space above in order to make sure this
   % \marginpar gets correctly placed.
   }%

%%%%%%%%%%%%%%%%%%%%%%%%%%%%%%%%%%%%%%%%%%%%%%%%%%%%%%%%%%%%%


%%%%%%%%%%%%%%%%%%%%%%%%%%%%%%%%%%%%%%%%%%%%%%%%%%%%%%%%%%%%%
% Make title
%\title{\vspace{2in}\textmd{\textbf{\hmwkClass:\ \hmwkTitle}}\\\normalsize\vspace{0.1in}\small{Due\ on\ \hmwkDueDate}\\\vspace{0.1in}\large{}\vspace{3in}}
\title{Tutorial: Plotting and Colors}
\date{}
%\author{\textbf{\hmwkAuthorName}}
%%%%%%%%%%%%%%%%%%%%%%%%%%%%%%%%%%%%%%%%%%%%%%%%%%%%%%%%%%%%%

\begin{document}
\begin{spacing}{1.0}
%\maketitle
%\newpage



\section{Tutorial: Plotting and Colors}

We're going to try to understand a little more about IDL's colors.  In the
process, we're going to do quite a bit of program editing work.

In order to cut down a bit on typing interactively at the command prompt,
I'm providing you with some functioning code.

\subsection{Get the code from git}

Change directories to your ASTR2600 git fork directory.

In that directory, do the following:

First, this command:\\
\verb|git remote add upstream git@github.com:ASTR2600f12/ASTR2600.git| \\
This crazily verbose command means:
``Add the ORIGINAL repository (that Adam made) as a `remote repository' in
my clone.  Call it `upstream'. ''

What is a remote repository?  It's a location on some other server you can
``push'' to (send your changes to) or ``pull'' from (get changes other people
have made).  

Why ``upstream''?  Because new code can flow from it, but you can't send your
changes directly to it.  

OK, why do this at all?  Because someone else - in this case, Adam - made a 
change to the original code, but your ``fork'' doesn't know about it yet.  So
you have to tell it to get the changes.

Next step:\\
\verb|git fetch upstream|

This will grab all of the changes from the remote (upstream) branch, but it
doesn't make any changes to your code at all yet.  Basically, this just
downloads information from the server.

\verb|git merge upstream/master|
Now you're ``merging'' the changes that I made with your files.  This WILL
make changes to your code!  “upstream" is the name of the remote repository,
“master" is the name of a “branch", which is something we haven't discussed;
we'll get there eventually.  For now, we always use master.

If it works, you should see something approximately like:
\begin{lstlisting}
Updating e2c3572..04e1e45
Fast-forward
 Tutorial13/color_ref.pro |   10 ++++++++++
 1 files changed, 10 insertions(+), 0 deletions(-)
 create mode 100644 Tutorial13/color_ref.pro
\end{lstlisting}    

The ``fast-forward'' bit means your code was out of date compared to upstream,
and you've now ``caught up''.  

\verb|Tutorial13/color_ref.pro| is a file that has changed (there may be
others, but this one should definitely be there).  The \verb|+|'s indicate how
many new lines have been added; since this file is completely new, all of the
lines are added.

Now, finally, you need to do:\\
\verb|git push| \\
to send the merge back to \emph{your} remote repository: you so far have only
made changes to the files on cosmos, and the download came from upstream, which
is not yours.

There is a lot more information on the push/pull model and all the tasks we
just did at this website:
\url{https://help.github.com/articles/using-pull-requests}, but I think much of
the language will be incomprehensible to you at this point.  

\subsection{Use the new code}
Open \verb|Tutorial13/color_ref.pro| in a text editor and keep it open.

Open IDL.  At the IDL prompt, run \verb|Tutorial13/color_ref.pro|.
For example, if you started IDL in the ASTR2600 fork directory, you'd do:\\
\verb|.r Tutorial13/color_ref.pro|

You have a color table!  But\dots what does it mean?  Looking at
\verb|color_ref.pro| probably doesn't tell you much.  There are no axis labels!
There are no comments!  OH NO!

Your job is to correct this!  

ASSIGNMENT: Correct \verb|color_ref.pro| to be more useful.  Make sure it is
adequately commented.  Commit your changes. When you have a
\verb|color_ref.pro| you're satisfied with, \verb|git push|  to send the
changes to the repository and then create a \emph{pull request} to suggest that
I accept your changes.  I'll give bonus points to whoever has the best
\verb|color_ref.pro|.

\subsection{Playing with Plots and Colors}
Next, run the \verb|eyeball.pro| script (note that this is a script, not
a program).  Also open it in a text editor.

Add comments to \verb|eyeball.pro| explaining what the unexplained lines (which
we identified in class) do.  

If there's anything you dislike about the so-called eyes, see if you can fix
them.  Possible fixes: 
\begin{enumerate}
    \item Convince IDL to draw within the lines 
    \item Make the eyes elliptical
    \item Add some other features frequently seen in eyes (e.g. whites)
\end{enumerate}

Finally, add a mouth!  This can probably be done with a parabola.  You
will also probably need to change the plot limits.  

Do this all within your personal \verb|eyeball.pro|.  Again, save and commit
your changes, and when you're done, push your changes and create a pull
request.

ASSIGNMENT: Comment, fix up (as much as you'd like, but at least a little),
and add a mouth to \verb|eyeball.pro|.

\end{spacing}
\end{document}

