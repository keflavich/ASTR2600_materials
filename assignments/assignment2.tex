%\documentclass[11pt,letterpaper,notitlepage,onesided]{tex/nwh_hw}
 %\documentclass[11pt,letterpaper,notitlepage]{article}
 \documentclass{article}
 % Change "article" to "report" to get rid of page number on title page
 \usepackage{amsmath,amsfonts,amsthm,amssymb}
 \usepackage{setspace}
 \usepackage{listings}
 \usepackage{Tabbing}
 \usepackage{textcomp}
 \usepackage{fancyhdr}
 \usepackage{lastpage}
 \usepackage{extramarks}
 \usepackage{chngpage}
 \usepackage{soul,color}
 \usepackage{graphicx,float,wrapfig}
 \usepackage{parskip}
 \usepackage[utf8]{inputenc}
 
 % In case you need to adjust margins:
 \topmargin=-0.45in      %
 \evensidemargin=0in     %
 \oddsidemargin=0in      %
 \textwidth=6.5in        %
 \textheight=9.0in       %
 \headsep=0.25in         %
 
 % Homework Specific Information
 \newcommand{\hmwkTitle}{Assignment 2: Spring 2012}
 \newcommand{\exDueDate}{September 11th, 4:00 PM}
 \newcommand{\hmwkDueDate}{September 13th, 11:59:59 PM}
 \newcommand{\hmwkClass}{ASTR 2600}
 \newcommand{\hmwkClassTime}{4:00 PM T/Th}
 \newcommand{\hmwkClassInstructor}{Adam Ginsburg}
 \newcommand{\hmwkAuthorName}{Adam Ginsburg}
 
 % Setup the header and footer
 \pagestyle{fancy}                                                       %
 %\lhead{\hmwkAuthorName}                                                 %
 \chead{\hmwkClass\: \hmwkTitle}  %
 \rhead{\firstxmark}                                                     %
 \lfoot{\lastxmark}                                                      %
 \cfoot{}                                                                %
 \rfoot{Page\ \thepage\ of\ \pageref{LastPage}}                          %
 \renewcommand\headrulewidth{0.4pt}                                      %
 \renewcommand\footrulewidth{0.4pt}                                      %
 
 \usepackage[utf8]{inputenc}
 \usepackage[unicode=true]{hyperref}
 \hypersetup{breaklinks=true,
             bookmarks=true,
             pdfauthor={},
             pdftitle={},
             colorlinks=true,
             urlcolor=blue,
             linkcolor=magenta,
             pdfborder={0 0 0}}
 
 % This is used to trace down (pin point) problems
 % in latexing a document:
 %\tracingall
 
 %%%%%%%%%%%%%%%%%%%%%%%%%%%%%%%%%%%%%%%%%%%%%%%%%%%%%%%%%%%%%
 % Some tools
 \newcommand{\enterProblemHeader}[1]{\nobreak\extramarks{#1}{#1 continued on next page\ldots}\nobreak%
                                     \nobreak\extramarks{#1 (continued)}{#1 continued on next page\ldots}\nobreak}%
 \newcommand{\exitProblemHeader}[1]{\nobreak\extramarks{#1 (continued)}{#1 continued on next page\ldots}\nobreak%
                                    \nobreak\extramarks{#1}{}\nobreak}%
 
 \newlength{\labelLength}
 \newcommand{\labelAnswer}[2]
   {\settowidth{\labelLength}{#1}%
    \addtolength{\labelLength}{0.25in}%
    \changetext{}{-\labelLength}{}{}{}%
    \noindent\fbox{\begin{minipage}[c]{\columnwidth}#2\end{minipage}}%
    \marginpar{\fbox{#1}}%
 
    % We put the blank space above in order to make sure this
    % \marginpar gets correctly placed.
    \changetext{}{+\labelLength}{}{}{}}%
 
 \setcounter{secnumdepth}{0}
 \newcommand{\homeworkProblemName}{}%
 \newcounter{homeworkProblemCounter}%
 \newenvironment{homeworkProblem}[1][Problem \arabic{homeworkProblemCounter}]%
   {\stepcounter{homeworkProblemCounter}%
    \renewcommand{\homeworkProblemName}{#1}%
    \section{\homeworkProblemName}%
    \enterProblemHeader{\homeworkProblemName}}%
   {\exitProblemHeader{\homeworkProblemName}}%
 
 \newcommand{\problemAnswer}[1]
   {\noindent\fbox{\begin{minipage}[c]{\columnwidth}#1\end{minipage}}}%
 
 \newcommand{\problemLAnswer}[1]
   {\labelAnswer{\homeworkProblemName}{#1}}
 
 \newcommand{\homeworkSectionName}{}%
 \newlength{\homeworkSectionLabelLength}{}%
 \newenvironment{homeworkSection}[1]%
   {% We put this space here to make sure we're not connected to the above.
    % Otherwise the changetext can do funny things to the other margin
 
    \renewcommand{\homeworkSectionName}{#1}%
    \settowidth{\homeworkSectionLabelLength}{\homeworkSectionName}%
    \addtolength{\homeworkSectionLabelLength}{0.25in}%
    \changetext{}{-\homeworkSectionLabelLength}{}{}{}%
    \subsection{\homeworkSectionName}%
    \enterProblemHeader{\homeworkProblemName\ [\homeworkSectionName]}}%
   {\enterProblemHeader{\homeworkProblemName}%
 
    % We put the blank space above in order to make sure this margin
    % change doesn't happen too soon (otherwise \sectionAnswer's can
    % get ugly about their \marginpar placement.
    \changetext{}{+\homeworkSectionLabelLength}{}{}{}}%
 
 \newcommand{\sectionAnswer}[1]
   {% We put this space here to make sure we're disconnected from the previous
    % passage
 
    \noindent\fbox{\begin{minipage}[c]{\columnwidth}#1\end{minipage}}%
    \enterProblemHeader{\homeworkProblemName}\exitProblemHeader{\homeworkProblemName}%
    \marginpar{\fbox{\homeworkSectionName}}%
 
    % We put the blank space above in order to make sure this
    % \marginpar gets correctly placed.
    }%
 
 %%%%%%%%%%%%%%%%%%%%%%%%%%%%%%%%%%%%%%%%%%%%%%%%%%%%%%%%%%%%%
 
 
 %%%%%%%%%%%%%%%%%%%%%%%%%%%%%%%%%%%%%%%%%%%%%%%%%%%%%%%%%%%%%
 % Make title
 \title{\vspace{2in}\textmd{\textbf{\hmwkClass:\ \hmwkTitle}}\\\normalsize\vspace{0.1in}\small{Due\ on\ \hmwkDueDate}\\\vspace{0.1in}\large{}\vspace{3in}}
 \date{}
 %\author{\textbf{\hmwkAuthorName}}
 %%%%%%%%%%%%%%%%%%%%%%%%%%%%%%%%%%%%%%%%%%%%%%%%%%%%%%%%%%%%%
 
 \begin{document}
 \begin{spacing}{1.0}
 %\maketitle
 \newpage
 
 
 
 \section{\textbf{Exercise:} \emph{  Due by classtime \exDueDate}}
 
  IDL data types, advanced array usage, coordinate arrays, numerical integration\\* 
 ~\\ 
 ~\\ 
 \textrm{\textrm{\textbf{Journal file:}}\textrm{\textbf{  }}}\verb@YourName_ex2.pro@\textrm{ }\textrm{  }\textrm{(in your }\verb@~/ASTR2600/assignment2/@\textrm{ directory)}\\* 
 ~\\ 
 \textbf{Exercise 2.0} Integer Data types \\* 
 \textbf{Exercise 2.1:} Floating point data types\\* 
 \textbf{Exercise 2.2:} String data type \& \verb@string@ function\\* 
 ~\\ 
 \textbf{Exercise 3.0:}  Array creation and element access\\* 
 \textbf{Exercise 3.1:} Array arithmetic, array functions, Vectors with Arrays\\* 
 \textbf{Exercise 3.2:} Rescaling a Coordinate Array\\* 
 \textbf{Exercise 3.3:} Random numbers to simulate experimental data\\* 
 \textbf{Exercise 3.4:} Numerical derivative from finite difference\\* 
 \textbf{Exercise 3.5:} Numerical integration of polynomial from x = -2. to 7. using array operations\\* 
 ~\\ 
 \textrm{\textrm{\textbf{Turn in}}}\textrm{ (copy into }\verb@/home/shared/astr2600/assignment2/YourName/@\textrm{):}\\* 
 Journal file of the exercise, \verb@YourName_ex2.pro@\\* 
 ~\\ 
 ~\\ 
 ~\\ 
 Answer all Whuduzitdos from Chapters 2 \& 3 in a file called \verb@YourName_wdid2.txt@.\\* 
 ~\\ 
 \textrm{\textrm{\textbf{Turn in}}}\textrm{ (copy into }\verb@/home/shared/astr2600/assignment2/YourName/@\textrm{)}\\* 
 Text file containing the Whuduzitdos:\verb@ YourName_wdid2.txt@\\* 
 ~\\ 
 (you can also hand in handwritten paper copies)
 ~\\ 
 ~\\ 
 \newpage \textbf{Graded Homework}\textbf{    }\textbf{Due by \hmwkDueDate}\\* 
 ~\\ 
 New journal files for each part.  \emph{Be careful to get the filenames right so you don’t overwrite!}\\* 
 That’s the normal process.\\* 
 ~\\ 
 \textrm{\textrm{\textbf{Homework 3.0:}}}\textrm{  }\textrm{Journal file: }\verb@assignment2/YourName_hw3.0.pro@\\* 
 \textrm{\textrm{\textbf{Homework 3.1:}}}\textrm{  }\textrm{Journal file: }\verb@assignment2/YourName_hw3.1.pro@\\* 
 \textrm{\textrm{\textbf{Homework 3.2:}}}\textrm{  }\textrm{Journal file: }\verb@assignment2/YourName_hw3.2.pro@\\* 
   Answer the questions as comments.\\* 
 \textrm{\textrm{\textbf{Homework 3.3:}}}\textrm{  }\textrm{Journal file: }\verb@Assignment2/YourName_hw3.3.pro@\\* 
 ~\\ 
 \textrm{\textrm{\textbf{Turn in}}}\textrm{ (copy into }\verb@/home/shared/astr2600/assignment2/YourName/@\textrm{:}\\* 
 Journal files of the homework:\\* 
 \verb@YourName_hw3.0.pro@\\* 
 \verb@YourName_hw3.1.pro@\\* 
 \verb@YourName_hw3.2.pro@\\* 
 \verb@YourName_hw3.3.pro@\\* 
 ~\\ 
 Note that the filenames have homework numbers of \verb@3@ but the turn-in directory is \verb@assignment2@.\\* 
 (this is because the \verb|3| refers to the chapter number, not the assignment number)
 ~\\ 
 ~\\ 
 ~\\ 
 
 
 \end{spacing}
 \end{document}
 
