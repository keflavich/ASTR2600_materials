\documentclass[12pt]{report}
\usepackage{nwh_lecture}
\usepackage{graphicx}
%\usepackage{graphics}
\usepackage{color}
%\usepackage{cite}      % bibliographic sitations
\usepackage{natbib}
%\citestyle{aa}
%\usepackage{floatflt}
\graphicspath{{figures/}}
\DeclareGraphicsExtensions{.pdf, .png}

% Macros, commands, definitions

\newcommand{\mycomment}[1]{{\bf\it\color{red} #1}} % Use for bold in-tex comments.


\newcommand{\clickerquestion}{[Clicker Question]}

% Math definitions
\newcommand{\abs}[1]{\ensuremath{\left| \mathbf{#1} \right|}}
\newcommand{\real}[1]{\mathrm{Re}\left( #1 \right)} 
\newcommand{\imag}[1]{\mathrm{Im}\left( #1 \right)} 
\newcommand{\ftdt}{\frac{\widetilde{\Delta T}}{T}}
%\newcommand{\exn}[1]{\left< #1 \right>} % expectation value.
\newcommand{\exn}[1]{\mathrm{E}\left[ #1 \right]} % expectation value.
\newcommand{\var}[1]{\mathrm{V}\left[ #1 \right]} % variance.
\newcommand{\tr}[1]{\mathrm{Tr}\left[ #1 \right]} % trace.
\newcommand{\uv}{\ensuremath{(u,v)}}
\newcommand{\su}{\ensuremath{\mathcal{S}(u)}}
\newcommand{\cl}{\ensuremath{\mathcal{C}_l}}
\newcommand{\vect}[1]{\ensuremath{\mathbf{#1}}}
\newcommand{\hour}{\ensuremath{^\mathrm{h}}}
\newcommand{\e}{\ensuremath{\operatorname{e}}}
\newcommand{\cov}[1]{\operatorname{COV}\left[ #1 \right]}
\newcommand{\FT}[1]{\operatorname{FT}\left\{ #1 \right\}} % expectation value.
\newcommand{\DFT}[1]{\operatorname{DFT}\left\{ #1 \right\}} % expectation value.

%%%%%% AAS defs %%%%%%%%%%%
\newcommand\fdg{\ensuremath{.\!\!^\circ}}
\newcommand\degr{\ensuremath{^\circ}}
\newcommand\arcmin{\mbox{$^\prime$}}% 
\newcommand\arcsec{\mbox{$^{\prime\prime}$}}% 
\newcommand\micron{\mbox{$\mu$m}}% 
\newcommand\tablenotemark[1]{\rlap{$^{\mathrm #1}$}}% 
\newcommand\tablenotetext[2]{\vspace{.5ex}{\footnotesize{\noindent\llap{$^{#1}$}#2\par}}} 


\makeatletter
\let\jnl@style=\rmfamily 
\def\ref@jnl#1{{\jnl@style#1}}% 
\newcommand\aj{\ref@jnl{AJ}}% 
          % Astronomical Journal 
\newcommand\araa{\ref@jnl{ARA\&A}}% 
          % Annual Review of Astron and Astrophys 
\newcommand\apj{\ref@jnl{ApJ}}% 
          % Astrophysical Journal 
\newcommand\apjl{\ref@jnl{ApJ}}% 
          % Astrophysical Journal, Letters 
\newcommand\apjs{\ref@jnl{ApJS}}% 
          % Astrophysical Journal, Supplement 
\newcommand\ao{\ref@jnl{Appl.~Opt.}}% 
          % Applied Optics 
\newcommand\apss{\ref@jnl{Ap\&SS}}% 
          % Astrophysics and Space Science 
\newcommand\aap{\ref@jnl{A\&A}}% 
          % Astronomy and Astrophysics 
\newcommand\aapr{\ref@jnl{A\&A~Rev.}}% 
          % Astronomy and Astrophysics Reviews 
\newcommand\aaps{\ref@jnl{A\&AS}}% 
          % Astronomy and Astrophysics, Supplement 
\newcommand\azh{\ref@jnl{AZh}}% 
          % Astronomicheskii Zhurnal 
\newcommand\baas{\ref@jnl{BAAS}}% 
          % Bulletin of the AAS 
\newcommand\jrasc{\ref@jnl{JRASC}}% 
          % Journal of the RAS of Canada 
\newcommand\memras{\ref@jnl{MmRAS}}% 
          % Memoirs of the RAS 
\newcommand\mnras{\ref@jnl{MNRAS}}% 
          % Monthly Notices of the RAS 
\newcommand\pra{\ref@jnl{Phys.~Rev.~A}}% 
          % Physical Review A: General Physics 
\newcommand\prb{\ref@jnl{Phys.~Rev.~B}}% 
          % Physical Review B: Solid State 
\newcommand\prc{\ref@jnl{Phys.~Rev.~C}}% 
          % Physical Review C 
\newcommand\prd{\ref@jnl{Phys.~Rev.~D}}% 
          % Physical Review D 
\newcommand\pre{\ref@jnl{Phys.~Rev.~E}}% 
          % Physical Review E 
\newcommand\prl{\ref@jnl{Phys.~Rev.~Lett.}}% 
          % Physical Review Letters 
\newcommand\pasp{\ref@jnl{PASP}}% 
          % Publications of the ASP 
\newcommand\pasj{\ref@jnl{PASJ}}% 
          % Publications of the ASJ 
\newcommand\qjras{\ref@jnl{QJRAS}}% 
          % Quarterly Journal of the RAS 
\newcommand\skytel{\ref@jnl{S\&T}}% 
          % Sky and Telescope 
\newcommand\solphys{\ref@jnl{Sol.~Phys.}}% 
          % Solar Physics 
\newcommand\sovast{\ref@jnl{Soviet~Ast.}}% 
          % Soviet Astronomy 
\newcommand\ssr{\ref@jnl{Space~Sci.~Rev.}}% 
          % Space Science Reviews 
\newcommand\zap{\ref@jnl{ZAp}}% 
          % Zeitschrift fuer Astrophysik 
\newcommand\nat{\ref@jnl{Nature}}% 
          % Nature 
\newcommand\iaucirc{\ref@jnl{IAU~Circ.}}% 
          % IAU Cirulars 
\newcommand\aplett{\ref@jnl{Astrophys.~Lett.}}% 
          % Astrophysics Letters 
\newcommand\apspr{\ref@jnl{Astrophys.~Space~Phys.~Res.}}% 
          % Astrophysics Space Physics Research 
\newcommand\bain{\ref@jnl{Bull.~Astron.~Inst.~Netherlands}}% 
          % Bulletin Astronomical Institute of the Netherlands 
\newcommand\fcp{\ref@jnl{Fund.~Cosmic~Phys.}}% 
          % Fundamental Cosmic Physics 
\newcommand\gca{\ref@jnl{Geochim.~Cosmochim.~Acta}}% 
          % Geochimica Cosmochimica Acta 
\newcommand\grl{\ref@jnl{Geophys.~Res.~Lett.}}% 
          % Geophysics Research Letters 
\newcommand\jcp{\ref@jnl{J.~Chem.~Phys.}}% 
          % Journal of Chemical Physics 
\newcommand\jgr{\ref@jnl{J.~Geophys.~Res.}}% 
          % Journal of Geophysics Research 
\newcommand\jqsrt{\ref@jnl{J.~Quant.~Spec.~Radiat.~Transf.}}% 
          % Journal of Quantitiative Spectroscopy and Radiative Trasfer 
\newcommand\memsai{\ref@jnl{Mem.~Soc.~Astron.~Italiana}}% 
          % Mem. Societa Astronomica Italiana 
\newcommand\nphysa{\ref@jnl{Nucl.~Phys.~A}}% 
          % Nuclear Physics A 
\newcommand\physrep{\ref@jnl{Phys.~Rep.}}% 
          % Physics Reports 
\newcommand\physscr{\ref@jnl{Phys.~Scr}}% 
          % Physica Scripta 
\newcommand\planss{\ref@jnl{Planet.~Space~Sci.}}% 
          % Planetary Space Science 
\newcommand\procspie{\ref@jnl{Proc.~SPIE}}% 
          % Proceedings of the SPIE



\begin{document}

\setcounter{chapter}{0}

\chapter{Course Overview}


Wed, 18 Jan 2012

%Reading for next class: Wall \& Jenkins, Ch. 1.

Today:
\begin{itemize}
\item Course Overview
  \begin{itemize}
  \item Course goals
    
  \item Syllabus overview
    
  \item My expectations and teaching philosophy
  \end{itemize}
\item Concept inventory (last 20 minutes of class)

\end{itemize}

\section{Course Overview}

Introductions and names

Course Goals:
\begin{itemize}
\item Understand principles and limitations of statistical inference and
  data analysis techniques

\item Understand the fundamentals of how instruments at multiple wavelengths
  collect and affect data
  
\item Derive physical measurements and uncertainties with hands-on analysis
  of real datasets. 
\end{itemize}

Course content and timeline: syllabus review.

Suggestions on topics relevant to your research are welcome.

\section{Introduction to Astronomical Statistics}

Motivation: Why is statistics important?
\begin{itemize}
\item Any observational science is one of probabilities.
  
\item Statistics is particularly important in astronomy, a data starved
  science:
  \begin{itemize}
  \item Observations \& instruments are pushed to the limits of
    detectability
    
  \item Small sample sizes, repeating the experiment not possible
    
  \item ``Cosmic variance'' --- we live in the one realization of the
    observable universe
  \end{itemize}
\item Examples of statistical inference in astronomy:
  \begin{itemize}
  \item Source detection (e.g, optical counterpart of a GRB)
    
  \item Significance of spectral features
    
  \item Statistical characterization of CMB temperature fluctuations
    
  \item Tests of correlations
    \begin{itemize}
    \item Hertzprung-Russell diagram
      
    \item Host galaxy characteristics as predictors for the presence of
      masers
      
    \item Hubble diagram
    \end{itemize}
  \item Hypothesis testing
    \begin{itemize}
    \item Isotropy of the universe
      
    \item Physical association of objects
      
    \item GRBs: galactic or extragalactic origin?
    \end{itemize}
  \item Model parameter estimation
    \begin{itemize}
    \item Estimating $\Omega_0$ from CMB measurements
      
    \item Dark matter distribution in galaxy halos
      
    \item Molecular abundance ratios in extragalactic clouds
    \end{itemize}
  \end{itemize}
\end{itemize}
In addition, statistics are ubiquitous in the astronomical literature and in
many other fields. They are used and abused frequently.

\emph{$\Rightarrow$ We must understand statistics to interpret results (ours
  and others), and to guard against naive belief in statistically questionable
  results. }

Examples of bad statistics:
\begin{itemize}
\item Medical study reports: drug is ineffective, \emph{except} in a small
  subpopulation.
  
\item Political race poll results: one candidate is ``ahead'' by 2 points,
  although only 1000 people are polled.
  
\item Astronomy: a statistically significant result is claimed, but
  systematic uncertainties are not accounted for.
\end{itemize}

[Concept Inventory]

\bibliographystyle{unsrt}
\bibliography{nwh}

\end{document}
