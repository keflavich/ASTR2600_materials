%\documentclass[11pt,letterpaper,notitlepage,onesided]{tex/nwh_hw}
 %\documentclass[11pt,letterpaper,notitlepage]{article}
 \documentclass{article}
 % Change "article" to "report" to get rid of page number on title page
 \usepackage{amsmath,amsfonts,amsthm,amssymb}
 \usepackage{setspace}
 \usepackage{listings}
 \usepackage{Tabbing}
 \usepackage{textcomp}
 \usepackage{fancyhdr}
 \usepackage{lastpage}
 \usepackage{extramarks}
 \usepackage{chngpage}
 \usepackage{soul,color}
 \usepackage{graphicx,float,wrapfig}
 \usepackage{parskip}
 \usepackage[utf8]{inputenc}
 
 % In case you need to adjust margins:
 \topmargin=-0.45in      %
 \evensidemargin=0in     %
 \oddsidemargin=0in      %
 \textwidth=6.5in        %
 \textheight=9.0in       %
 \headsep=0.25in         %
 
 % Homework Specific Information
 \newcommand{\hmwkTitle}{Assignment 5: Programming, \texttt{IF..THEN},  \texttt{CASE}, \texttt{FOR}, \texttt{WHILE} loops, dynamical modelling}
 \newcommand{\exDueDate}{October 2th, 4:00 PM}
 \newcommand{\hmwkDueDate}{October 4th, 11:59:59 PM}
 \newcommand{\hmwkClass}{ASTR 2600}
 \newcommand{\hmwkClassTime}{4:00 PM T/Th}
 \newcommand{\hmwkClassInstructor}{Adam Ginsburg}
 \newcommand{\hmwkAuthorName}{Adam Ginsburg}
 
 % Setup the header and footer
 \pagestyle{fancy}                                                       %
 %\lhead{\hmwkAuthorName}                                                 %
 \chead{\hmwkClass\: \hmwkTitle}  %
 \rhead{\firstxmark}                                                     %
 \lfoot{\lastxmark}                                                      %
 \cfoot{}                                                                %
 \rfoot{Page\ \thepage\ of\ \pageref{LastPage}}                          %
 \renewcommand\headrulewidth{0.4pt}                                      %
 \renewcommand\footrulewidth{0.4pt}                                      %
 
 \usepackage[utf8]{inputenc}
 \usepackage[unicode=true]{hyperref}
 \hypersetup{breaklinks=true,
             bookmarks=true,
             pdfauthor={},
             pdftitle={},
             colorlinks=true,
             urlcolor=blue,
             linkcolor=magenta,
             pdfborder={0 0 0}}
 
 % This is used to trace down (pin point) problems
 % in latexing a document:
 %\tracingall
 
 %%%%%%%%%%%%%%%%%%%%%%%%%%%%%%%%%%%%%%%%%%%%%%%%%%%%%%%%%%%%%
 % Some tools
 \newcommand{\enterProblemHeader}[1]{\nobreak\extramarks{#1}{#1 continued on next page\ldots}\nobreak%
                                     \nobreak\extramarks{#1 (continued)}{#1 continued on next page\ldots}\nobreak}%
 \newcommand{\exitProblemHeader}[1]{\nobreak\extramarks{#1 (continued)}{#1 continued on next page\ldots}\nobreak%
                                    \nobreak\extramarks{#1}{}\nobreak}%
 
 \newlength{\labelLength}
 \newcommand{\labelAnswer}[2]
   {\settowidth{\labelLength}{#1}%
    \addtolength{\labelLength}{0.25in}%
    \changetext{}{-\labelLength}{}{}{}%
    \noindent\fbox{\begin{minipage}[c]{\columnwidth}#2\end{minipage}}%
    \marginpar{\fbox{#1}}%
 
    % We put the blank space above in order to make sure this
    % \marginpar gets correctly placed.
    \changetext{}{+\labelLength}{}{}{}}%
 
 \setcounter{secnumdepth}{0}
 \newcommand{\homeworkProblemName}{}%
 \newcounter{homeworkProblemCounter}%
 \newenvironment{homeworkProblem}[1][Problem \arabic{homeworkProblemCounter}]%
   {\stepcounter{homeworkProblemCounter}%
    \renewcommand{\homeworkProblemName}{#1}%
    \section{\homeworkProblemName}%
    \enterProblemHeader{\homeworkProblemName}}%
   {\exitProblemHeader{\homeworkProblemName}}%
 
 \newcommand{\problemAnswer}[1]
   {\noindent\fbox{\begin{minipage}[c]{\columnwidth}#1\end{minipage}}}%
 
 \newcommand{\problemLAnswer}[1]
   {\labelAnswer{\homeworkProblemName}{#1}}
 
 \newcommand{\homeworkSectionName}{}%
 \newlength{\homeworkSectionLabelLength}{}%
 \newenvironment{homeworkSection}[1]%
   {% We put this space here to make sure we're not connected to the above.
    % Otherwise the changetext can do funny things to the other margin
 
    \renewcommand{\homeworkSectionName}{#1}%
    \settowidth{\homeworkSectionLabelLength}{\homeworkSectionName}%
    \addtolength{\homeworkSectionLabelLength}{0.25in}%
    \changetext{}{-\homeworkSectionLabelLength}{}{}{}%
    \subsection{\homeworkSectionName}%
    \enterProblemHeader{\homeworkProblemName\ [\homeworkSectionName]}}%
   {\enterProblemHeader{\homeworkProblemName}%
 
    % We put the blank space above in order to make sure this margin
    % change doesn't happen too soon (otherwise \sectionAnswer's can
    % get ugly about their \marginpar placement.
    \changetext{}{+\homeworkSectionLabelLength}{}{}{}}%
 
 \newcommand{\sectionAnswer}[1]
   {% We put this space here to make sure we're disconnected from the previous
    % passage
 
    \noindent\fbox{\begin{minipage}[c]{\columnwidth}#1\end{minipage}}%
    \enterProblemHeader{\homeworkProblemName}\exitProblemHeader{\homeworkProblemName}%
    \marginpar{\fbox{\homeworkSectionName}}%
 
    % We put the blank space above in order to make sure this
    % \marginpar gets correctly placed.
    }%
 
 %%%%%%%%%%%%%%%%%%%%%%%%%%%%%%%%%%%%%%%%%%%%%%%%%%%%%%%%%%%%%
 
 
 %%%%%%%%%%%%%%%%%%%%%%%%%%%%%%%%%%%%%%%%%%%%%%%%%%%%%%%%%%%%%
 % Make title
 \title{\vspace{2in}\textmd{\textbf{\hmwkClass:\ \hmwkTitle}}\\\normalsize\vspace{0.1in}\small{Due\ on\ \hmwkDueDate}\\\vspace{0.1in}\large{}\vspace{3in}}
 \date{}
 %\author{\textbf{\hmwkAuthorName}}
 %%%%%%%%%%%%%%%%%%%%%%%%%%%%%%%%%%%%%%%%%%%%%%%%%%%%%%%%%%%%%
 
 \begin{document}
 \begin{spacing}{1.0}
 %\maketitle
 %\newpage
 
 
 
 \section{\textbf{Exercise:} \emph{  Due by classtime \exDueDate}}
 
 As usual, do all of this in your \verb@assignment5/@ directory.  
 ~\\
~\\ 
 ~\\
~\\ 
 \textbf{Exercise 10.0}: Creating a script file to set the values of physical constants \& units  \\
 Open a journal file, \verb@YourName_ex4_10.0.pro@, to do the interactive part. 

 \textbf{Exercise 10.1:} Creating an interactive script file  \\
 Open a journal file, \verb@YourName_ex4_10.1.pro@, to do the interactive part. 

 \textbf{Exercise 10.2:} Contour plotting using a script  \\
 Last line on page 22 says you should name the JPEG file as
 \verb@ex10_2_plot.jpg@.  You should, of course, stick \verb@YourName_@ on the
 front of that.) \\
 Open a journal file, \verb@YourName_ex4_10.2.pro@, to do the interactive part. 
 ~\\
~\\ 
 \textbf{Exercise 11.0:} Interactive \emph{program} (like interactive \emph{script} in 10.1). \\
   A good starting point would be to make a copy of your \verb@plot_Planck_ch10.pro@ file from 10.1. \\
 \textbf{Exercise 12.0}: Using \verb@IF..THEN..ELSE@ blocks (in a \verb@FOR@ loop) \\
 \textbf{Exercise 12.1}: Interactive programming example. \\
 \textbf{Exercise 12.2}:\verb@ WHILE@ loop exercise \\
 \textbf{Exercise 12.3:} Simple 1D dynamics with \verb@FOR@ loop \\
 ~\\
~\\ 
 \textbf{Turn in} all \verb@.pro@ files you write for the exercise to the
 \verb@assignment5@ directory: \\
 \verb@/home/shared/astr2600/assignment5/YourName@  \\
 \verb@YourName_plot_Planck_ch11.pro@  \\
 \verb@YourName_piecewise.pro@  \\
 \verb@YourName_plot_Plancks_ch12.pro@  \\
 \verb@YourName_factorial.pro@  \\
 \verb@YourName_spring1D.pro @\verb@    @\verb@(last version)@  \\
 ~\\
~\\ 
 \textbf{Whuduzitdo?}  Do all the Whuduzitdo’s for Chapters 11 \& 12.  Filename: \verb@YourName_wdid5.txt@ 
 ~\\
~\\ 
\textbf{Tutorial 13}  Complete the edits to \verb|eyeball.pro| and
\verb|color_ref.pro| and create a pull request on github to turn them in.
 ~\\
~\\ 
 \newpage \textbf{Graded Homework}\textbf{  }Due by \hmwkDueDate 
 ~\\
~\\ 
 \textbf{Homework 12.0:} Interactive Planck program.  Filename: \verb@YourName_plot_Planck_hw12.0.pro@  \\
 \textbf{Homework 12.1:} Interactive function plotter. Filename: \verb@YourName_plot_functions_hw12.1.pro@ \\ 
 ~\\
~\\ 
 Three springs: These homeworks each deal with the same objective, modeling 3
 springs.  The first is an expansion of Exercise 12.3.  Each of the subsequent
 parts modifies the one before.  The resulting behavior should always be the
 same, so you can check your program at each stage by making sure it is.  \\
 \textbf{Homework}\textbf{  }\textbf{12.2:} Three springs: filename \verb@YourName_springs_hw12.2.pro@  \\
 \textbf{Homework 12.3:} Use arrays of spring constants.   
 There is no separate 12.3.  Just 12.3a, 12.3b, etc.  Each version should be in
 its own file, e.g., \verb@YourName_springs_hw12.3a.pro@, etc. \\
 \textbf{  }\textbf{Homework 12.3a:} Array of initial conditions, scalars for current conditions  \\
 \textbf{  }\textbf{Homework 12.3b:} Array of initial conditions, arrays for current conditions  \\
 \textbf{  }\textbf{Homework 12.3c:} Swap loops to put spring loop inside time loop  \\
 \textbf{  }\textbf{Homework 12.3d:} Replace a spring loop with array arithmetic 
   
 ~\\
~\\ 
 \textbf{Turn in} all \verb@.pro@ files you write for the homework to the \verb@assignment5@ directory: \\
 \verb@/home/shared/astr2600/assignment5/YourName/@ \\
 \verb@YourName_plot_Planck_hw12.0.pro@ \\
 \verb@YourName_plot_functions_hw12.1.pro@ \\
 \verb@YourName_springs_hw12.2.pro@ \\
 \verb@YourName_springs_hw12.3a.pro@ \\
 \verb@YourName_springs_hw12.3b.pro@ \\
 \verb@YourName_springs_hw12.3c.pro@ \\
 \verb@YourName_springs_hw12.3d.pro@  \\
 ~\\
~\\ 
 ~\\
~\\ 
 ~\\
~\\ 
 ~\\
~\\ 
 ~\\
~\\ 
 ~\\
~\\ 
 ~\\
~\\ 
 ~\\
~\\ 
 ~\\
~\\ 
 ~\\
~\\ 
 
 
 \end{spacing}
 \end{document}
 
