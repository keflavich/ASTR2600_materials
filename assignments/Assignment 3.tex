%\documentclass[11pt,letterpaper,notitlepage,onesided]{tex/nwh_hw}
%\documentclass[11pt,letterpaper,notitlepage]{article}
\documentclass{article}
% Change "article" to "report" to get rid of page number on title page
\usepackage{amsmath,amsfonts,amsthm,amssymb}
\usepackage{setspace}
\usepackage{listings}
\usepackage{Tabbing}
\usepackage{textcomp}
\usepackage{fancyhdr}
\usepackage{lastpage}
\usepackage{extramarks}
\usepackage{chngpage}
\usepackage{soul,color}
\usepackage{graphicx,float,wrapfig}
\usepackage{parskip}
\usepackage[utf8]{inputenc}

% In case you need to adjust margins:
\topmargin=-0.45in      %
\evensidemargin=0in     %
\oddsidemargin=0in      %
\textwidth=6.5in        %
\textheight=9.0in       %
\headsep=0.25in         %

% Homework Specific Information
\newcommand{\hmwkTitle}{Assignment 3: file reading, binary files, sort, where, Booleans
}
\newcommand{\hmwkDueDate}{DATE, 4:00 PM}
\newcommand{\hmwkClass}{ASTR 2600}
\newcommand{\hmwkClassTime}{4:00 PM T/Th}
\newcommand{\hmwkClassInstructor}{Adam Ginsburg}
\newcommand{\hmwkAuthorName}{Dewey Anderson}

% Setup the header and footer
\pagestyle{fancy}                                                       %
%\lhead{\hmwkAuthorName}                                                 %
\chead{\hmwkClass\: \hmwkTitle}  %
\rhead{\firstxmark}                                                     %
\lfoot{\lastxmark}                                                      %
\cfoot{}                                                                %
\rfoot{Page\ \thepage\ of\ \pageref{LastPage}}                          %
\renewcommand\headrulewidth{0.4pt}                                      %
\renewcommand\footrulewidth{0.4pt}                                      %

% This is used to trace down (pin point) problems
% in latexing a document:
%\tracingall

%%%%%%%%%%%%%%%%%%%%%%%%%%%%%%%%%%%%%%%%%%%%%%%%%%%%%%%%%%%%%
% Some tools
\newcommand{\enterProblemHeader}[1]{\nobreak\extramarks{#1}{#1 continued on next page\ldots}\nobreak%
                                    \nobreak\extramarks{#1 (continued)}{#1 continued on next page\ldots}\nobreak}%
\newcommand{\exitProblemHeader}[1]{\nobreak\extramarks{#1 (continued)}{#1 continued on next page\ldots}\nobreak%
                                   \nobreak\extramarks{#1}{}\nobreak}%

\newlength{\labelLength}
\newcommand{\labelAnswer}[2]
  {\settowidth{\labelLength}{#1}%
   \addtolength{\labelLength}{0.25in}%
   \changetext{}{-\labelLength}{}{}{}%
   \noindent\fbox{\begin{minipage}[c]{\columnwidth}#2\end{minipage}}%
   \marginpar{\fbox{#1}}%

   % We put the blank space above in order to make sure this
   % \marginpar gets correctly placed.
   \changetext{}{+\labelLength}{}{}{}}%

\setcounter{secnumdepth}{0}
\newcommand{\homeworkProblemName}{}%
\newcounter{homeworkProblemCounter}%
\newenvironment{homeworkProblem}[1][Problem \arabic{homeworkProblemCounter}]%
  {\stepcounter{homeworkProblemCounter}%
   \renewcommand{\homeworkProblemName}{#1}%
   \section{\homeworkProblemName}%
   \enterProblemHeader{\homeworkProblemName}}%
  {\exitProblemHeader{\homeworkProblemName}}%

\newcommand{\problemAnswer}[1]
  {\noindent\fbox{\begin{minipage}[c]{\columnwidth}#1\end{minipage}}}%

\newcommand{\problemLAnswer}[1]
  {\labelAnswer{\homeworkProblemName}{#1}}

\newcommand{\homeworkSectionName}{}%
\newlength{\homeworkSectionLabelLength}{}%
\newenvironment{homeworkSection}[1]%
  {% We put this space here to make sure we're not connected to the above.
   % Otherwise the changetext can do funny things to the other margin

   \renewcommand{\homeworkSectionName}{#1}%
   \settowidth{\homeworkSectionLabelLength}{\homeworkSectionName}%
   \addtolength{\homeworkSectionLabelLength}{0.25in}%
   \changetext{}{-\homeworkSectionLabelLength}{}{}{}%
   \subsection{\homeworkSectionName}%
   \enterProblemHeader{\homeworkProblemName\ [\homeworkSectionName]}}%
  {\enterProblemHeader{\homeworkProblemName}%

   % We put the blank space above in order to make sure this margin
   % change doesn't happen too soon (otherwise \sectionAnswer's can
   % get ugly about their \marginpar placement.
   \changetext{}{+\homeworkSectionLabelLength}{}{}{}}%

\newcommand{\sectionAnswer}[1]
  {% We put this space here to make sure we're disconnected from the previous
   % passage

   \noindent\fbox{\begin{minipage}[c]{\columnwidth}#1\end{minipage}}%
   \enterProblemHeader{\homeworkProblemName}\exitProblemHeader{\homeworkProblemName}%
   \marginpar{\fbox{\homeworkSectionName}}%

   % We put the blank space above in order to make sure this
   % \marginpar gets correctly placed.
   }%

%%%%%%%%%%%%%%%%%%%%%%%%%%%%%%%%%%%%%%%%%%%%%%%%%%%%%%%%%%%%%


%%%%%%%%%%%%%%%%%%%%%%%%%%%%%%%%%%%%%%%%%%%%%%%%%%%%%%%%%%%%%
% Make title
\title{\vspace{2in}\textmd{\textbf{\hmwkClass:\ \hmwkTitle}}\\\normalsize\vspace{0.1in}\small{Due\ on\ \hmwkDueDate}\\\vspace{0.1in}\large{}\vspace{3in}}
\date{}
%\author{\textbf{\hmwkAuthorName}}
%%%%%%%%%%%%%%%%%%%%%%%%%%%%%%%%%%%%%%%%%%%%%%%%%%%%%%%%%%%%%

\begin{document}
\begin{spacing}{1.0}
%\maketitle
\newpage



\section{\textbf{Exercise:} \emph{  Due by classtime \hmwkDueDate}}


\textbf{ASTR 2600 Assignment 3} ~ formatted printing, file printing,



\textbf{Journal file\emph{s}:~}\verb|yourName\_ex3\_\#.\#.pro|~ where \#.\# is
actually the number of the exercise the file begins with, e.g., 4.0.~
There are a lot of parts to this exercise so you should split it into at
least two different journal files.~ Rather than tell you where to split
it up, I'll let you pick what feels convenient.~ The only requirement is
that you make the split between book exercises, i.e., not in the middle
of one.

You should do this in your \textasciitilde{}/ASTR2600/assignment3/
directory.


\textbf{Exercise 4.0}: Optional parameter in plot, using max.

\textbf{Exercise 5.0}: Play around with some formatted printing

\textbf{Exercise 6.0}: Writing a text file.~ (Repeats last part of
Exercise 5.0 but writes output to a file.)~

This says to modify the file with a text editor.~ Use gedit for that.~
(Reminder: Open another terminal window.~ At the UNIX prompt, cd into
your assignment3 directory.~ Then run gedit by typing gedit
powerTable.txt.~ This gets you running the gedit text editor, with the
powerTable.txt file.)

\textbf{Exercise 6.1}: Reading the text file created in Exercise 6.0.

\textbf{Exercise 6.2}: Binary files

This has you compare two versions of output.~ You may need to make your
terminal window taller by resizing it with the mouse to see both
printouts.

It then has you ``use the operating system'' to compare the sizes of two
files.~ Use ls --l in UNIX to see the sizes of the files.~ (Use a
separate window).

\textbf{}\\

\textbf{Exercise 6.4 (Not in the book):} Saving and restoring an IDL
save file.


help ~ ~ ~ ~ ; Shows the list of variables IDL is currently aware of (if
any)

.reset\_session~ ; Tells IDL to forget all those variables (\& procs \&
funcs)

help~ ~ ~ ~ ~ ~ ; Verify that IDL has forgotten all those variables


; create an array of x values from --10 to 10

x = -10 + 20.*findgen(100)/99.

y = sin(x) / x ~ ~ ~ ~ ~ ~ ~ ~ ~ ; sin(x)/x is called a ``sinc''
function

help ~ ~ ~ ~ ~ ~ ~ ~ ~ ~ ~ ~ ~ ~ ; See that you now have x \& y arrays
defined

plot, x, y ~ ~ ~ ~ ~ ~ ~ ~ ~ ~ ~ ; Take a look at it


; Save the x \& y variables in an IDL save file:

save, x, y, filename=``sinc.idlsave''


.reset\_session~ ~ ; Tell IDL to forget all about those

help~ ~ ~ ~ ~ ~ ~ ; Verify that IDL has forgotten all those variables



; restore the variables from the file

restore, filename=``sinc.idlsave''

help~ ~ ~ ~ ~ ~ ~ ~ ~ ~ ~ ~ ; See that IDL now knows x \& y arrays

plot, x, y~ ~ ~ ~ ~ ~ ~ ~ ~ ; Plot them


\textbf{}\\

\textbf{Exercise 6.3}: Saving an image file. ~

If you'd like, you can just restore the file from Exercise 6.4 instead
of redefining x \& y arrays.

You can view this jpg file by using the window file browser in Linux.~
Work your way down to your ASTR2600/assignment3 directory and click on
the yourName\_mySincPlot.jpg ~ ~ It should come up in an image viewing
program.





\textbf{Exercise 6.5 (Not in the book)}:~ Working with a file whose
``endian'' convention is different.

We have two data files, one with x values, the other with y values.

Each contains data in a ``self-describing'' format.~ The first data in
each file is a long integer that says how many floats follow it.~ The
file then has that many floats.~ A convenient way to say this is ``It
contains a long integer, N, followed by N floats.''

The files are \textasciitilde{}andersdk/ASTR2600/assignment3/sinc\_x.dat
and \textasciitilde{}andersdk/ASTR2600/assignment3/sinc\_y.dat

We need to open both, read N for each, then read in the x and y values.

Then plot, x, y


; open BOTH data files

; We'll do this using some string operators.

; This is just to go through those motions. ~

; We could just define the filenames in the open statements

path = ``\textasciitilde{}andersdk/ASTR2600/assignment3/''~ ~ ~ ; the
path is something the filenames share

xfilename = path + ``sinc\_x.dat''~ ~ ~ ~ ; build the full filename with
path

yfilename = path + ``sinc\_y.dat''

print, xfilename ~ ~ ~ ~ ~ ~ ~ ~ ~ ~ ~ ; verify that the filenames are
right

print, yfilename

openr, xlun, xfilename, /get\_lun ~ ; Open the files.~ Notice that this
LUN is xlun

openr, ylun, yfilename, /get\_lun ~ ; and this LUN is ylun


; Now read in nx and ny from the two files

nx = 0L~ ~ ~ ~ ~ ~ ~ ~ ~ ~ ~ ~ ~ ~ ~ ; ``declare'' nx and ny to be
longs.

ny = 0L~ ~ ~ ~ ~ ~ ~ ~ ~ ~ ~ ~ ~ ~ ~ ; (It doesn't matter what actual
value we use)

readu, xlun, nx~ ~ ~ ~ ~ ~ ~ ~ ~ ~ ~ ; read in the number of x values

readu, ylun, ny~ ~ ~ ~ ~ ~ ~ ~ ~ ~ ~ ; read in the number of y values

~~ ~ ~ ~ ~ ~ ~ ~ ~ ~ ~ ~ ~ ~ ~ ~ ~ ~ ; we expect nx is equal to ny


x = fltarr(nx) ~ ~ ~ ~ ~ ~ ~ ~ ~ ~ ~ ; create the array for x

~~ ~ ~ ~ ~ ~ ~ ~ ~ ~ ~ ~ ~ ~ ~ ~ ~ ~ ; OOPS!~ Something's wrong!

print, nx~ ~ ~ ~ ~ ~ ~ ~ ~ ~ ~ ~ ~ ~ ; check the nx value


; It makes no sense that nx is negative.

; This is our clue that the datafile is big-endian and we are
little-endian

; (or the other way around).~ Whichever case, we need to swap the bytes
in nx


nx = swap\_endian(nx)~ ~ ~ ~ ~ ; replaces nx with a byte-swapped version
of nx

print, nx ~ ~ ~ ~ ~ ~ ~ ~ ~ ~ ; Does this value make more sense?

print, ny ~ ~ ~ ~ ~ ~ ~ ~ ~ ~ ; verify that ny is similarly goofy

ny = swap\_endian(ny)~ ~ ~ ~ ~ ; so also need to swap bytes for ny


x = fltarr(nx)~ ~ ~ ~ ~ ~ ~ ; NOW we can create our arrays with sensible
numbers

y = fltarr(ny)

readu, xlun, x~ ~ ~ ~ ~ ~ ~ ; Read in the actual arrays of numbers

readu, ylun, y ~ ~ ~ ~ ~ ~ ~


free\_lun, xlun~ ~ ~ ~ ~ ~ ~ ; We're done reading the files so we clos
them

free\_lun, ylun


x = swap\_endian(x)~ ~ ~ ~ ~ ; x and y must ALSO be byte-swapped

y = swap\_endian(y)


plot, x, y~ ~ ~ ~ ~ ~ ~ ~ ~ ; plot a nice sinc function



\textbf{Exercise 7.0}: A sorting exercise.

\textbf{}\\

\textbf{Exercise 7.1}: Comparisons \& Booleans.

\textbf{}\\

\textbf{Exercise 7.2}: Using where with comparisons \& Booleans.

\textbf{Exercise 7.3}: Finding where the minimum is using where.

You've now seen 3 ways to find a minimum (or maximum): Use the min \&
max functions, use where, and compute the derivative and see where it
crosses 0.





\textbf{Turn in} (copy into
\textasciitilde{}andersdk/ASTR2600/students/assignment3/yourName/)

All journal files of the exercises, \verb|yourName\_ex3\_X.X.pro|

testArray.txt~ ~ (These files do not have to have your name in their
name.)

testArray.dat

JPEG image file from Exercise 6.3: yourName\_mySincPlot.jpg




\textbf{Whuduzitdo:} All Whuduzitdo's from Chapters 4-7.

\textbf{Graded Homework ~}Due by midnight the night of Monday, Feb 20,
2012

Do all of this in your ASTR2600/assignment3 directory


\textbf{Homework 4.0:} Journal file: \verb|yourName\_hw4.0.pro|.~ Finding peak
of Planck function using max.~

\textbf{Homework 6.0:} Journal file: \verb|yourName\_hw6.0.pro|.~ Writing
binary file.


~You don't need to do Homeworks 6.1 \& 6.2 \& 6.3


\textbf{Homework 6.4 (Not in book):} Journal file: \verb|yourName\_hw6.4.pro|

In the directory \textasciitilde{}andersdk/ASTR2600/assignment3/ there
is a file, plotData.dat

~(That's not a typo: There is no ``students'' in the path.)

It's a binary file containing a \emph{long} integer, say, N, followed by
an array of N floats.

Read this file and plot the array. ~

(\emph{Don't} copy the file into your local directory and read that.~
Include the full directory path in the filename when you open the file.~
It's not always practical to make local copies of data files.~ You need
to know how to read in files from other directories.)

Since I don't give you an x array, you can assume evenly spaced values,
so ``plot, y'' works.

What is your guess as to the kind of function I used to generate it?~
(Hint: Think math, not physics.)

Write your response as an IDL comment so that it appears in your journal
file, e.g. ~


; I think it was another damn Planck function


(Don't worry.~ It's OK to be wrong.)


\textbf{Homework 7.1 (Not in book):} Journal file: \verb|yourName\_hw7.1.pro|

In the \textasciitilde{}andersdk/ASTR2600/assignment3 directory there
are two files: lif2bwave.dat and lif2bflux.dat.~ These are binary files
containing ultraviolet spectra data from a star.

lif2bwave.dat contains the wavelengths in angstroms.

lif2bflux.dat contains the flux in ergs/cm\textsuperscript{2}

The format of these binary files is a long integer, N, followed by N
floats.

a) Read these files in and plot the flux vs. wavelength.~ (Again, don't
make local copies of the files.~ Read them in from where they sit.)

~ Save an image of this plot as \textbf{yourName\_fullSpectrum.jpg}

b) The plot has a double peak near 990 angstroms. ~

~ Plot about a 5-angstrom range roughly centered on this double peak so
that we see it clearly.

~ (Use the where function to select the portions of the wave and flux
arrays that fit this.~ This is like ``zooming in'' on that part of the
spectrum.)

~ Save an image of this plot as \textbf{yourName\_doublePeak.jpg}


\textbf{Turn in} (copy into
\textasciitilde{}andersdk/ASTR2600/students/assignment3/yourName/)

Journal files of the homework:

\verb|yourName\_hw4.0.pro|

\verb|yourName\_hw6.0.pro|

\verb|yourName\_hw6.4.pro|~

\verb|yourName\_hw7.1.pro|~

Image files:\textbf{~}

yourName\_fullSpectrum.jpg

yourName\_doublePeak.jpg




\end{spacing}
\end{document}

