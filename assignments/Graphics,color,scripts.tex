%\documentclass[11pt,letterpaper,notitlepage,onesided]{tex/nwh_hw}
 %\documentclass[11pt,letterpaper,notitlepage]{article}
 \documentclass{article}
 % Change "article" to "report" to get rid of page number on title page
 \usepackage{amsmath,amsfonts,amsthm,amssymb}
 \usepackage{setspace}
 \usepackage{listings}
 \usepackage{Tabbing}
 \usepackage{textcomp}
 \usepackage{fancyhdr}
 \usepackage{lastpage}
 \usepackage{extramarks}
 \usepackage{chngpage}
 \usepackage{soul,color}
 \usepackage{graphicx,float,wrapfig}
 \usepackage{parskip}
 \usepackage[utf8]{inputenc}
 
 % In case you need to adjust margins:
 \topmargin=-0.45in      %
 \evensidemargin=0in     %
 \oddsidemargin=0in      %
 \textwidth=6.5in        %
 \textheight=9.0in       %
 \headsep=0.25in         %
 
 % Homework Specific Information
 \newcommand{\hmwkTitle}{Graphics, color, scripts: Assignment 4}
 \newcommand{\hmwkDueDate}{DATE, 4:00 PM}
 \newcommand{\hmwkClass}{ASTR 2600}
 \newcommand{\hmwkClassTime}{4:00 PM T/Th}
 \newcommand{\hmwkClassInstructor}{Adam Ginsburg}
 \newcommand{\hmwkAuthorName}{Dewey Anderson}
 
 % Setup the header and footer
 \pagestyle{fancy}                                                       %
 %\lhead{\hmwkAuthorName}                                                 %
 \chead{\hmwkClass\: \hmwkTitle}  %
 \rhead{\firstxmark}                                                     %
 \lfoot{\lastxmark}                                                      %
 \cfoot{}                                                                %
 \rfoot{Page\ \thepage\ of\ \pageref{LastPage}}                          %
 \renewcommand\headrulewidth{0.4pt}                                      %
 \renewcommand\footrulewidth{0.4pt}                                      %
 
 % This is used to trace down (pin point) problems
 % in latexing a document:
 %\tracingall
 
 %%%%%%%%%%%%%%%%%%%%%%%%%%%%%%%%%%%%%%%%%%%%%%%%%%%%%%%%%%%%%
 % Some tools
 \newcommand{\enterProblemHeader}[1]{\nobreak\extramarks{#1}{#1 continued on next page\ldots}\nobreak%
                                     \nobreak\extramarks{#1 (continued)}{#1 continued on next page\ldots}\nobreak}%
 \newcommand{\exitProblemHeader}[1]{\nobreak\extramarks{#1 (continued)}{#1 continued on next page\ldots}\nobreak%
                                    \nobreak\extramarks{#1}{}\nobreak}%
 
 \newlength{\labelLength}
 \newcommand{\labelAnswer}[2]
   {\settowidth{\labelLength}{#1}%
    \addtolength{\labelLength}{0.25in}%
    \changetext{}{-\labelLength}{}{}{}%
    \noindent\fbox{\begin{minipage}[c]{\columnwidth}#2\end{minipage}}%
    \marginpar{\fbox{#1}}%
 
    % We put the blank space above in order to make sure this
    % \marginpar gets correctly placed.
    \changetext{}{+\labelLength}{}{}{}}%
 
 \setcounter{secnumdepth}{0}
 \newcommand{\homeworkProblemName}{}%
 \newcounter{homeworkProblemCounter}%
 \newenvironment{homeworkProblem}[1][Problem \arabic{homeworkProblemCounter}]%
   {\stepcounter{homeworkProblemCounter}%
    \renewcommand{\homeworkProblemName}{#1}%
    \section{\homeworkProblemName}%
    \enterProblemHeader{\homeworkProblemName}}%
   {\exitProblemHeader{\homeworkProblemName}}%
 
 \newcommand{\problemAnswer}[1]
   {\noindent\fbox{\begin{minipage}[c]{\columnwidth}#1\end{minipage}}}%
 
 \newcommand{\problemLAnswer}[1]
   {\labelAnswer{\homeworkProblemName}{#1}}
 
 \newcommand{\homeworkSectionName}{}%
 \newlength{\homeworkSectionLabelLength}{}%
 \newenvironment{homeworkSection}[1]%
   {% We put this space here to make sure we're not connected to the above.
    % Otherwise the changetext can do funny things to the other margin
 
    \renewcommand{\homeworkSectionName}{#1}%
    \settowidth{\homeworkSectionLabelLength}{\homeworkSectionName}%
    \addtolength{\homeworkSectionLabelLength}{0.25in}%
    \changetext{}{-\homeworkSectionLabelLength}{}{}{}%
    \subsection{\homeworkSectionName}%
    \enterProblemHeader{\homeworkProblemName\ [\homeworkSectionName]}}%
   {\enterProblemHeader{\homeworkProblemName}%
 
    % We put the blank space above in order to make sure this margin
    % change doesn't happen too soon (otherwise \sectionAnswer's can
    % get ugly about their \marginpar placement.
    \changetext{}{+\homeworkSectionLabelLength}{}{}{}}%
 
 \newcommand{\sectionAnswer}[1]
   {% We put this space here to make sure we're disconnected from the previous
    % passage
 
    \noindent\fbox{\begin{minipage}[c]{\columnwidth}#1\end{minipage}}%
    \enterProblemHeader{\homeworkProblemName}\exitProblemHeader{\homeworkProblemName}%
    \marginpar{\fbox{\homeworkSectionName}}%
 
    % We put the blank space above in order to make sure this
    % \marginpar gets correctly placed.
    }%
 
 %%%%%%%%%%%%%%%%%%%%%%%%%%%%%%%%%%%%%%%%%%%%%%%%%%%%%%%%%%%%%
 
 
 %%%%%%%%%%%%%%%%%%%%%%%%%%%%%%%%%%%%%%%%%%%%%%%%%%%%%%%%%%%%%
 % Make title
 \title{\vspace{2in}\textmd{\textbf{\hmwkClass:\ \hmwkTitle}}\\\normalsize\vspace{0.1in}\small{Due\ on\ \hmwkDueDate}\\\vspace{0.1in}\large{}\vspace{3in}}
 \date{}
 %\author{\textbf{\hmwkAuthorName}}
 %%%%%%%%%%%%%%%%%%%%%%%%%%%%%%%%%%%%%%%%%%%%%%%%%%%%%%%%%%%%%
 
 \begin{document}
 \begin{spacing}{1.0}
 %\maketitle
 \newpage
 
 
 
 \section{\textbf{Exercise:} \emph{  Due by classtime \hmwkDueDate}}
 
 \\ 
  \par \\ 
  \textbf{Exercise} :   Due by class time Wednesday, Feb 22, 2012 \\ 
  \par \\ 
  \verb|Journal file:| \textbf{yourName\_ex4.pro} \verb|(in your| \textasciitilde{}/ASTR2600/assignment4/ \verb|directory)| \\ 
  (Create an assignment4 subdirectory under your ASTR2600 directory if necessary.) \\ 
  Exercises from Chapter 8 and Chapter 9 involve journal files.   This is the last assignment with long journal files as exercises.   It’s probably a good idea to break this up into multiple journal files as described in Assignment 3. \\ 
  \par \\ 
  \textbf{Exercise 8.0} : Experimenting with various plot keywords \\ 
  \textbf{Exercise 8.1:}  Title \&  \verb|linestyle|  keywords, using  \verb|xyouts|  for legends \\ 
  \textbf{Exercise 8.2:}  A little zooming \\ 
  \textbf{Exercise 8.3:}  2D plotting \\ 
  \textbf{Exercise 8.4:}  Multiple plots \\ 
  \textbf{Exercise 9.0:}  Playing with color definitions. \\ 
  \textbf{Exercise 9.1} : Like Exercise 8.1 but with colors \\ 
  (You don’t have to do Exercise 9.2.) \\ 
  \par \\ 
  For Chapter 10 exercises you create script files and execute them instead of creating journal files.   Be sure to put  \verb|yourName_|  on the beginning of the file name specified for the script. \\ 
  But after creating your script file, you’ll actually  \emph{use}  it interactively.   You  \emph{open a journal file}  to do the interactive part. \\ 
  You can use  \verb|gedit|  as the text editor to create the scripts. \\ 
  \par \\ 
  \textbf{Exercise 10.0} : Creating a script file to set the values of physical constants \& units \\ 
  Open a journal file,  \verb|yourName_ex4_10.0.pro| , to do the interactive part. \\ 
  \textbf{Exercise 10.1:}  Creating an interactive script file \\ 
  Open a journal file,  \verb|yourName_ex4_10.1.pro| , to do the interactive part. \\ 
  \textbf{Exercise 10.2:}  Contour plotting using a script \\ 
  Last line on page 22 says you should name the JPEG file as  \verb|ex10_2_plot.jpg| .   You should, of course, stick  \verb|yourName_|  on the front of that.) \\ 
  Open a journal file,  \verb|yourName_ex4_10.2.pro| , to do the interactive part. \\ 
  \par \\ 
  \par \\ 
  \par \\ 
  \textbf{Whuduzitdo?}   All Whuduzitdo’s from Chapters 8, 9 \& 10. \\ 
  \par \\ 
  \par \\ 
  \verb|Turn in| \textasciitilde{}andersdk/ASTR2600/students/assignment4/yourName/ \verb|)| \\ 
  \textbf{Journal files}  of the exercises from Chapters 8 \& 9. \\ 
  Your IDL script files:   \\ 
  \textbf{yourName\_physicalConstants.pro} \\ 
  \textbf{yourName\_plot\_Planck\_ch10.pro} \\ 
  \textbf{yourName\_sinSinContour.pro} \\ 
  Your JPG file from Exercise 10.2,  \verb|yourName\_ex10_2_plot.jpg| \\ 
  \par \\ 
  \par \\ 
  \textbf{Graded Homework} Due by midnight the night of Monday, Feb 27, 2012    \\ 
  (Readability is 10\% of your grade) \\ 
  Homeworks 8.2 \& 8.3 should be done by writing  \emph{script}  files, not by doing them interactively and creating a journal file. \\ 
  \par \\ 
  \textbf{Homework 8.2:}  Script file name:  \verb|yourName_sinSinGrid.pro| \\ 
  \textbf{Homework 8.3:}  Script file name:  \verb|yourName_PlanckShaded.pro| \\ 
  \par \\ 
  \textbf{Homework 8.4 (Not in book):}  Script file,  \verb|yourName_plot_spectrum.pro|  that does the following: \\ 
  Read in the data files used in Homework 6.5 in assignment 3.   (Read them from my  \verb|assignment3|  directory (include path).   Do not copy them into yours.) \\ 
  Draw the following plots and save them into JPEG files as described.   The plots should all have  \emph{appropriate x \& y axes labels and an appropriate title} . \\ 
  If the yaxis label appears clipped off the left edge of the window, use the  \verb|xmargin|  keyword to remedy this.   (Read about it in the book or online documentation.) \\ 
  a) Plot the flux vs. wavelength.   (Déjà vu.   You did this in Homework 6.4) \\ 
  Save an image of this plot as  \verb|yourName_fullSpectrum.jpg| \\ 
  b) The plot has a double peak near 990 angstroms.    \\ 
  Edit your script to use the  \verb|xrange|  keyword (rather than the  \verb|where|  function used in Homework 6.4) to plot about a 5-angstrom range roughly centered on this double peak so that we see it clearly.   (This is like “zooming in” on that part of the spectrum.) \\ 
  Save an image of this plot as  \verb|yourName_doublePeak.jpg| \\ 
  c) The plot has a large, thin peak near 1026 angstroms. \\ 
  Edit the script to use the range keywords to plot about a 1-angstrom range roughly centered on this double peak.   This plot should also have a flux range of about  \verb|1e-12|  flux units so that we can clearly see the structure at this peak.   (This means setting a  \verb|yrange|  as well.) \\ 
  Save an image of this plot as  \verb|yourName_tallPeak.jpg| . \\ 
  \par \\ 
  \textbf{Homework 9.2 (Not in book)} : Create a script file  \verb|yourName_solarTemp.pro|  that does the following: \\ 
  Read a binary 500 \verb|x| 500 array of floats from the file: \\ 
  The format is simply the array of floats.   \\ 
    There is no first-number “header” with the array size.   I’m telling you the size.   It’s 500  \verb|x|  500. \\ 
  Display that array as a monochrome  \emph{image} \\ 
  Open another IDL plotting window (with the “ \verb|window| ” procedure) \\ 
  Plot the data as a color contour map with 10 contours and with the  \verb|/FILL|  keyword set. \\ 
    Define a range of colors where the largest contour values are the brightest.    \\ 
    Use multiple hues, e.g. blue to red to yellow \\ 
  The title of the plot should be “Solar Temperature” \\ 
  Save this contour plot as a JPG file with the name  \verb|yourName_solarTemp.jpg| \\ 
  \par \\ 
  \verb|Turn in| \textasciitilde{}andersdk/ASTR2600/students/assignment4/yourName/ \verb|)| \\ 
  Your IDL script files:   \\ 
  \textbf{yourName\_sinSinGrid.pro} \\ 
  \textbf{yourName\_PlanckShaded.pro} \\ 
  \textbf{yourName\_plot\_spectrum.pro}   (for  \emph{one}  of the plots) \\ 
  \textbf{yourName\_solarTemp.pro} \\ 
  All 4 of your jpg files \\ 
  \textbf{yourName\_fullSpectrum.jpg} \\ 
  \textbf{yourName\_doublePeak.jpg} \\ 
  \textbf{yourName\_tallPeak.jpg} \\ 
  \textbf{yourName\_solarTemp.jpg} \\ 
 
 
 \end{spacing}
 \end{document}
 
