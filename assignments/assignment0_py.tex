%\documentclass[11pt,letterpaper,notitlepage,onesided]{tex/nwh_hw}
%\documentclass[11pt,letterpaper,notitlepage]{article}
\documentclass{article}
% Change "article" to "report" to get rid of page number on title page
\usepackage{amsmath,amsfonts,amsthm,amssymb}
\usepackage{setspace}
\usepackage{listings}
\usepackage{Tabbing}
\usepackage{textcomp}
\usepackage{fancyhdr}
\usepackage{lastpage}
\usepackage{extramarks}
\usepackage{chngpage}
\usepackage{soul,color}
\usepackage{graphicx,float,wrapfig}
\usepackage{parskip}
\usepackage[utf8]{inputenc}

% In case you need to adjust margins:
\topmargin=-0.45in      %
\evensidemargin=0in     %
\oddsidemargin=0in      %
\textwidth=6.5in        %
\textheight=9.0in       %
\headsep=0.25in         %

% Homework Specific Information
\newcommand{\hmwkTitle}{Assignment 0:   Basic interactive python math}
\newcommand{\hmwkDueDate}{September 4th, 4:00 PM}
\newcommand{\hmwkClass}{ASTR 2600}
\newcommand{\hmwkClassTime}{4:00 PM T/Th}
\newcommand{\hmwkClassInstructor}{Adam Ginsburg}
\newcommand{\hmwkAuthorName}{Dewey Anderson}

% Setup the header and footer
\pagestyle{fancy}                                                       %
%\lhead{\hmwkAuthorName}                                                 %
\chead{\hmwkClass\: \hmwkTitle}  %
\rhead{\firstxmark}                                                     %
\lfoot{\lastxmark}                                                      %
\cfoot{}                                                                %
\rfoot{Page\ \thepage\ of\ \pageref{LastPage}}                          %
\renewcommand\headrulewidth{0.4pt}                                      %
\renewcommand\footrulewidth{0.4pt}                                      %

% This is used to trace down (pin point) problems
% in latexing a document:
%\tracingall

%%%%%%%%%%%%%%%%%%%%%%%%%%%%%%%%%%%%%%%%%%%%%%%%%%%%%%%%%%%%%
% Some tools
\newcommand{\enterProblemHeader}[1]{\nobreak\extramarks{#1}{#1 continued on next page\ldots}\nobreak%
                                    \nobreak\extramarks{#1 (continued)}{#1 continued on next page\ldots}\nobreak}%
\newcommand{\exitProblemHeader}[1]{\nobreak\extramarks{#1 (continued)}{#1 continued on next page\ldots}\nobreak%
                                   \nobreak\extramarks{#1}{}\nobreak}%

\newlength{\labelLength}
\newcommand{\labelAnswer}[2]
  {\settowidth{\labelLength}{#1}%
   \addtolength{\labelLength}{0.25in}%
   \changetext{}{-\labelLength}{}{}{}%
   \noindent\fbox{\begin{minipage}[c]{\columnwidth}#2\end{minipage}}%
   \marginpar{\fbox{#1}}%

   % We put the blank space above in order to make sure this
   % \marginpar gets correctly placed.
   \changetext{}{+\labelLength}{}{}{}}%

\setcounter{secnumdepth}{0}
\newcommand{\homeworkProblemName}{}%
\newcounter{homeworkProblemCounter}%
\newenvironment{homeworkProblem}[1][Problem \arabic{homeworkProblemCounter}]%
  {\stepcounter{homeworkProblemCounter}%
   \renewcommand{\homeworkProblemName}{#1}%
   \section{\homeworkProblemName}%
   \enterProblemHeader{\homeworkProblemName}}%
  {\exitProblemHeader{\homeworkProblemName}}%

\newcommand{\problemAnswer}[1]
  {\noindent\fbox{\begin{minipage}[c]{\columnwidth}#1\end{minipage}}}%

\newcommand{\problemLAnswer}[1]
  {\labelAnswer{\homeworkProblemName}{#1}}

\newcommand{\homeworkSectionName}{}%
\newlength{\homeworkSectionLabelLength}{}%
\newenvironment{homeworkSection}[1]%
  {% We put this space here to make sure we're not connected to the above.
   % Otherwise the changetext can do funny things to the other margin

   \renewcommand{\homeworkSectionName}{#1}%
   \settowidth{\homeworkSectionLabelLength}{\homeworkSectionName}%
   \addtolength{\homeworkSectionLabelLength}{0.25in}%
   \changetext{}{-\homeworkSectionLabelLength}{}{}{}%
   \subsection{\homeworkSectionName}%
   \enterProblemHeader{\homeworkProblemName\ [\homeworkSectionName]}}%
  {\enterProblemHeader{\homeworkProblemName}%

   % We put the blank space above in order to make sure this margin
   % change doesn't happen too soon (otherwise \sectionAnswer's can
   % get ugly about their \marginpar placement.
   \changetext{}{+\homeworkSectionLabelLength}{}{}{}}%

\newcommand{\sectionAnswer}[1]
  {% We put this space here to make sure we're disconnected from the previous
   % passage

   \noindent\fbox{\begin{minipage}[c]{\columnwidth}#1\end{minipage}}%
   \enterProblemHeader{\homeworkProblemName}\exitProblemHeader{\homeworkProblemName}%
   \marginpar{\fbox{\homeworkSectionName}}%

   % We put the blank space above in order to make sure this
   % \marginpar gets correctly placed.
   }%

%%%%%%%%%%%%%%%%%%%%%%%%%%%%%%%%%%%%%%%%%%%%%%%%%%%%%%%%%%%%%


%%%%%%%%%%%%%%%%%%%%%%%%%%%%%%%%%%%%%%%%%%%%%%%%%%%%%%%%%%%%%
% Make title
\title{\vspace{2in}\textmd{\textbf{\hmwkClass:\ \hmwkTitle}}\\\normalsize\vspace{0.1in}\small{Due\ on\ \hmwkDueDate}\\\vspace{0.1in}\large{}\vspace{3in}}
\date{}
%\author{\textbf{\hmwkAuthorName}}
%%%%%%%%%%%%%%%%%%%%%%%%%%%%%%%%%%%%%%%%%%%%%%%%%%%%%%%%%%%%%

\begin{document}
\begin{spacing}{1.0}
%\maketitle
\newpage


\section{\textbf{Exercise:} \emph{  Due by classtime \hmwkDueDate}}
\par

\noindent Do this in your home login directory in Unix.

\noindent Run ipython and type each courier font line exactly as it appears
below.  You do not have to type the “comment” consisting of the semi-colon and
all text after it.  Pay attention to the output.  Each line illustrates some
different behavior.

\noindent The first thing you will do, of course, is open a terminal window and run ipython by typing

\texttt{ipython}

at the Linux prompt.  Now you’re running ipython and you should see the ipython prompt.

Unlike IDL, python does not have `journaling' - instead, each ipython session
will automatically log commands and output.  Your commands will be logged in a
file called \verb|ipython_log_2012-08-23.py|, although the date will be
replaced with the date you started the interactive session.

There is another way to run ipython using something called a `notebook'.  If
you start ipython with the command \verb|ipython notebook --pylab &|, a new tab
will pop open in your browser with buttons to make a \verb|New Notebook| and
open your saved notebooks.  This is the preferred method for using ipython in
this class, as it makes it very easy to record your work and all the figures
you make along the way.

Now proceed with these exercises from the book:

\begin{itemize}
    \item Exercise 1.0 Basic arithmetic
    \item Exercise 1.1 Basic use of variables
    \item Exercise 1.2 Operator precedence
    \item Exercise 1.3 String arithmetic
    \item Exercise 1.4 Math functions
    \item Exercise 1.5 Equations
    \item Exercise 1.6 A good physics-based complicated equation, the Planck black-body curve
    \item Exercise 1.7 Modifying a variable
\end{itemize}

Before each exercise, type a comment or note that labels the exercise.  
Comments must be preceded by the \verb|#| symbol, e.g.

\texttt{# Exercise 1.2}

In the Notebook, comments can be added by creating ``Markdown'' cells
(``markdown'' indicates that you can write in special notation if you want to
write equations).

Remember to save your notebook frequently.

When you are done with ipython, you exit it with \texttt{exit()} or close the
ipython notebook window (after saving!).

If for some reason you do not finish all the exercises in one sitting, you can
close the notebook file and then, later, reopen it and continue working. 


\section{\textbf{What does it do?} }
{\it This section describes how to edit a .txt file for the ``What does it do?'' 
assignment; you are allowed to include the ``What does it do?'' component in the
ipython notebook if you prefer.}

From the terminal window Linux prompt, run the \texttt{gedit} text editor on a
file named \texttt{YourName\_wdid0.txt}.  To do this, at the Linux prompt, type
the following:

\texttt{gedit YourName\_wdid0.txt \&}

(I’ll explain the “\texttt{\&}”, called an “ampersand”, later in the semester.)
\texttt{gedit} is a typical text editor similar in principle to word processors
you’ve used.  Don’t use a word processor like MS Word to do this.  You need to
learn to use the text editor in Linux\footnote{You can use \texttt{vi} or
\texttt{emacs} if you want.  I'll help you with \texttt{vi}, but not with
\texttt{emacs}}.   This will open the new (or existing)
file allowing you to edit it.  

In this file, type your answers to the \emph{What does it do?} 1.0 through 1.7.
If there is no explicit question, then the implied question is “What does it
do?”, e.g., if the last statement is a print statement, “What does it print?”

When you are done, use the \texttt{gedit} menus to save the file and quit the editor.

If you do not finish the \emph{What does it do?} in one sitting, you can save
the file and quit.  Then, when you are ready to finish it, you can edit the
same file (type the same command).  This will allow you to append the later
answers.  You’ll never need to turn in \emph{What does it do?}’s in multiple
pieces.


\section{Turn in}
copy into \texttt{\textasciitilde ginsbura/ASTR2600/students/assignment0/YourName/}:
\begin{tabbing}
Notebook file of the exercise, \texttt{YourName\_ex0.ipynb} \\
\emph{What does it do?} text file: \texttt{YourName\_wdid0.txt}
\end{tabbing}

To do this, in Unix, type

\begin{tabbing}
\texttt{cp -p YourName\_ex0.ipynb \textasciitilde ginsbura/ASTR2600/students/assignment0/yourName/} \\
\texttt{cp -p YourName\_wdid0.txt \textasciitilde ginsbura/ASTR2600/students/assignment0/yourName/}
\end{tabbing}

There is no Graded Homework for Assignment 0.
(This is why I let you have a week to do the exercise \& \emph{What does it do?}.)

\end{spacing}
\end{document}
