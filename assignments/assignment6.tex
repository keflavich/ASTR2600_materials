%\documentclass[11pt,letterpaper,notitlepage,onesided]{tex/nwh_hw}
 %\documentclass[11pt,letterpaper,notitlepage]{article}
 \documentclass{article}
 % Change "article" to "report" to get rid of page number on title page
 \usepackage{amsmath,amsfonts,amsthm,amssymb}
 \usepackage{setspace}
 \usepackage{listings}
 \usepackage{Tabbing}
 \usepackage{textcomp}
 \usepackage{fancyhdr}
 \usepackage{lastpage}
 \usepackage{extramarks}
 \usepackage{chngpage}
 \usepackage{soul,color}
 \usepackage{graphicx,float,wrapfig}
 \usepackage{parskip}
 \usepackage[compact]{titlesec}
 \usepackage[utf8]{inputenc}
 
 % In case you need to adjust margins:
 \topmargin=-0.45in      %
 \evensidemargin=0in     %
 \oddsidemargin=0in      %
 \textwidth=6.5in        %
 \textheight=9.0in       %
 \headsep=0.25in         %
 
 % Homework Specific Information
 \newcommand{\hmwkTitle}{Assignment 6: Procedures and functions, 2D dynamics}
 \newcommand{\hmwkDueDate}{October 9th, 4:00 PM}
 \newcommand{\hmwkClass}{ASTR 2600}
 \newcommand{\hmwkClassTime}{4:00 PM T/Th}
 \newcommand{\hmwkClassInstructor}{Adam Ginsburg}
 \newcommand{\hmwkAuthorName}{Adam Ginsburg}
 
 % Setup the header and footer
 \pagestyle{fancy}                                                       %
 %\lhead{\hmwkAuthorName}                                                 %
 \chead{\hmwkClass\: \hmwkTitle}  %
 \rhead{\firstxmark}                                                     %
 \lfoot{\lastxmark}                                                      %
 \cfoot{}                                                                %
 \rfoot{Page\ \thepage\ of\ \pageref{LastPage}}                          %
 \renewcommand\headrulewidth{0.4pt}                                      %
 \renewcommand\footrulewidth{0.4pt}                                      %
 
 % This is used to trace down (pin point) problems
 % in latexing a document:
 %\tracingall
 
 %%%%%%%%%%%%%%%%%%%%%%%%%%%%%%%%%%%%%%%%%%%%%%%%%%%%%%%%%%%%%
 % Some tools
 \newcommand{\enterProblemHeader}[1]{\nobreak\extramarks{#1}{#1 continued on next page\ldots}\nobreak%
                                     \nobreak\extramarks{#1 (continued)}{#1 continued on next page\ldots}\nobreak}%
 \newcommand{\exitProblemHeader}[1]{\nobreak\extramarks{#1 (continued)}{#1 continued on next page\ldots}\nobreak%
                                    \nobreak\extramarks{#1}{}\nobreak}%
 
 \newlength{\labelLength}
 \newcommand{\labelAnswer}[2]
   {\settowidth{\labelLength}{#1}%
    \addtolength{\labelLength}{0.25in}%
    \changetext{}{-\labelLength}{}{}{}%
    \noindent\fbox{\begin{minipage}[c]{\columnwidth}#2\end{minipage}}%
    \marginpar{\fbox{#1}}%
 
    % We put the blank space above in order to make sure this
    % \marginpar gets correctly placed.
    \changetext{}{+\labelLength}{}{}{}}%
 
 \setcounter{secnumdepth}{0}
 \newcommand{\homeworkProblemName}{}%
 \newcounter{homeworkProblemCounter}%
 \newenvironment{homeworkProblem}[1][Problem \arabic{homeworkProblemCounter}]%
   {\stepcounter{homeworkProblemCounter}%
    \renewcommand{\homeworkProblemName}{#1}%
    \section{\homeworkProblemName}%
    \enterProblemHeader{\homeworkProblemName}}%
   {\exitProblemHeader{\homeworkProblemName}}%
 
 \newcommand{\problemAnswer}[1]
   {\noindent\fbox{\begin{minipage}[c]{\columnwidth}#1\end{minipage}}}%
 
 \newcommand{\problemLAnswer}[1]
   {\labelAnswer{\homeworkProblemName}{#1}}
 
 \newcommand{\homeworkSectionName}{}%
 \newlength{\homeworkSectionLabelLength}{}%
 \newenvironment{homeworkSection}[1]%
   {% We put this space here to make sure we're not connected to the above.
    % Otherwise the changetext can do funny things to the other margin
 
    \renewcommand{\homeworkSectionName}{#1}%
    \settowidth{\homeworkSectionLabelLength}{\homeworkSectionName}%
    \addtolength{\homeworkSectionLabelLength}{0.25in}%
    \changetext{}{-\homeworkSectionLabelLength}{}{}{}%
    \subsection{\homeworkSectionName}%
    \enterProblemHeader{\homeworkProblemName\ [\homeworkSectionName]}}%
   {\enterProblemHeader{\homeworkProblemName}%
 
    % We put the blank space above in order to make sure this margin
    % change doesn't happen too soon (otherwise \sectionAnswer's can
    % get ugly about their \marginpar placement.
    \changetext{}{+\homeworkSectionLabelLength}{}{}{}}%
 
 \newcommand{\sectionAnswer}[1]
   {% We put this space here to make sure we're disconnected from the previous
    % passage
 
    \noindent\fbox{\begin{minipage}[c]{\columnwidth}#1\end{minipage}}%
    \enterProblemHeader{\homeworkProblemName}\exitProblemHeader{\homeworkProblemName}%
    \marginpar{\fbox{\homeworkSectionName}}%
 
    % We put the blank space above in order to make sure this
    % \marginpar gets correctly placed.
    }%
 
 %%%%%%%%%%%%%%%%%%%%%%%%%%%%%%%%%%%%%%%%%%%%%%%%%%%%%%%%%%%%%
 
 
 %%%%%%%%%%%%%%%%%%%%%%%%%%%%%%%%%%%%%%%%%%%%%%%%%%%%%%%%%%%%%
 % Make title
 \title{\vspace{2in}\textmd{\textbf{\hmwkClass:\ \hmwkTitle}}\\\normalsize\vspace{0.1in}\small{Due\ on\ \hmwkDueDate}\\\vspace{0.1in}\large{}\vspace{3in}}
 \date{}
 %\author{\textbf{\hmwkAuthorName}}
 %%%%%%%%%%%%%%%%%%%%%%%%%%%%%%%%%%%%%%%%%%%%%%%%%%%%%%%%%%%%%
 
 \begin{document}
 \begin{spacing}{1.0}
 %\maketitle
 \newpage
 
 
 
 \section{\textbf{Exercise:} \emph{  Due by classtime \hmwkDueDate}}
 
 \textbf{Exercise 13.0}: Filename \verb@yourName_plot_sinc.pro@.\\* 
 Do the interactive part in Open a \emph{journal} file, \verb@yourName_ex13.0.pro@.\\* 
 ~\\ 
 \textbf{Exercise 13.1: }Filename \verb@yourName_swap.pro@.  \emph{Include all comments}\\* 
 Do the interactive part in Open a \emph{journal} file, \verb@yourName_ex13.1.pro@.\\* 
 ~\\ 
 \textbf{Whuduzitdo?} for chapter 13.  Filename: \verb@yourName_wdid13.txt@\\* 
 ~\\ 
 \textbf{Turn in} to the directory   \verb@/home/shared/astr2600/assignment6/YourName/@\\* 
 \verb@yourName_plot_sinc.pro@\\* 
 \verb@yourName_ex13.0.pro@\\* 
 \verb@yourName_swap.pro@\\* 
 \verb@yourName_ex13.1.pro@\\* 
 \verb@yourName_wdid13.txt@

 \section{Homework 6: \emph{Due 11:59:59 pm Thursday, October 11th, 2012}}

 \textbf{Homework 14.0:} Filename \verb@yourName_coordinateArray.pro@\\* 
 Where you are asked to change the code (i.e., delete parts), you should
 \verb|git commit| before making the change.
 ~\\ 
 \textbf{Homework 14.1:} Filename: \verb@yourName_planck.pro@\\* 
 This file will contain a function, \verb@planck@, and a procedure called \verb@plot_Planck@ that will call it.\\* 
 \verb@plot_Planck@ will also call your \verb@coordinateArray@ function from Homework 14.0.\\* 
 That should mean compiling both files.  Don’t \emph{copy} the \verb@coordinateArray@ function into the file for this homework.\\* 
 ~\\ 
 \textbf{Homework 14.2:} Filename: \verb@yourName_twoD_2stars_HW14.2.pro@\\* 
 Binary stars.  This HW synthesize many of the previous chapters. \\* 
 ~\\ 
 \textbf{Homework 14.3:} Filename: \verb@yourName_twoD_2stars_HW14.3.pro@\\* 
 Modularize Homework 14.2.\\* 
 ~\\ 
 \textbf{Turn in} to the directory  \verb@ /home/shared/astr2600/assignment6/YourName/@\\* 
 \emph{also} turn in on github:
 \begin{lstlisting}
     git add [filename.pro]
     git commit -a
     git push
 \end{lstlisting}
 \verb@yourName_coordinateArray.pro@\\* 
 \verb@yourName_planck.pro@\\* 
 \verb@yourName_planck10000K.jpg@\\* 
 \verb@yourName_twoD_2stars_HW14.2.pro@\\* 
 \verb@yourName_twoD_2stars_HW14.3.pro@

 \section{Tutorial 14: Due 11:59:59 pm, October 11th 2012}
 Turn in via github:\\
 \verb|isthenumberbig.pro|\\
 \verb|compare_values.pro|\\
 \verb|test_conditionals.pro|\\
 
 
 \end{spacing}
 \end{document}
 
